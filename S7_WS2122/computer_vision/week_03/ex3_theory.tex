\documentclass[11pt]{article}
\usepackage{geometry}
\geometry{a4paper, top=20mm, left=10mm, right=10mm, bottom=20mm}
\usepackage{graphicx}
\usepackage{amsmath,amssymb,amsthm}
\usepackage{amssymb}
\usepackage{mathtools}
\usepackage[utf8]{inputenc}
\usepackage{fancyhdr}
\usepackage{lastpage}
\usepackage{enumerate}
\usepackage{enumitem}
\usepackage{multicol}
\usepackage{subcaption}
\usepackage{color}
\usepackage{scalerel}
%------------------------------------------ preamble
%----- fancyhdr
\fancyhead[L]{Name: Maurice Wenig}
\fancyhead[R]{Matrikelnummer: 178049}
\fancyfoot{}
\rfoot{Seite \thepage\ von \pageref{LastPage}}
\pagestyle{fancy}
%----- aufgaben
\newtheoremstyle{break}{}{5mm}{}{}{\bfseries}{}{0mm}
{\textbf{\thmname{#1}\thmnumber{ #2:} \thmnote{\textit{#3}}\newline}}
\theoremstyle{break}
\newtheorem{task}{Aufgabe}
%----- new commands
\newcommand{\Romannumeral}[1]{\MakeUppercase{\romannumeral #1}}
\newcommand{\set}[1]{\{#1\}}
\newcommand{\abs}[1]{\left\vert #1 \right\vert}
\newcommand{\norm}[1]{\left\| #1 \right\|}
\newcommand{\skal}[2]{\left\langle #1 | #2 \right\rangle}
\newcommand{\expected}[1]{\mathbb{E}\set{#1}}
\newcommand{\variance}[1]{\text{Var}\set{#1}}
%----- defs
\def\notiff{\mathrel{{\ooalign{\hidewidth$\not\phantom{"}$\hidewidth\cr$\iff$}}}}
\def\R{\ensuremath{\mathbb{R}}}
\def\1{\ensuremath{\mathbb{1}}}
\def\ndy{
    \textcolor{red} {\hfill not done yet!}
    \reversemarginpar
    \marginpar{\raggedleft\textcolor{red}{\rule{2mm}{2mm}}}
}
\def\fourier{\mathcal{F}}
\def\INT{\int_{-\infty}^{\infty}}
%----- homework number
\newcommand{\hw}{3}
%------------------------------------------ main
\begin{document}
%----- title
\begin{center}
    \Large{Rechnersehen Theorieaufgaben}\\
    \large{\hw. Übungsserie}
\end{center}
%----- tasks
\begin{task}
    \begin{align*}
        \shortintertext{
            $$g(x, y) = f(x, y) \cdot (-1)^{x + y}$$
        }
        \fourier(g)(u, v) & = \sum\limits_{x = 1}^{M}\sum\limits_{y = 1}^{N} f(x, y) \cdot (-1)^{x + y} e^{-2\pi i\left(\frac{ux}{M} + \frac{vy}{N}\right)}              \\
                          & = \sum\limits_{x = 1}^{M}\sum\limits_{y = 1}^{N} f(x, y) \cdot e^{i\pi(x + y)} \cdot e^{-2\pi i\left(\frac{ux}{M} + \frac{vy}{N}\right)}     \\
                          & = \sum\limits_{x = 1}^{M}\sum\limits_{y = 1}^{N} f(x, y) e^{-2\pi i\left(\frac{ux}{M} + \frac{vy}{N} - \frac{x + y}{2}\right)}               \\
                          & = \sum\limits_{x = 1}^{M}\sum\limits_{y = 1}^{N} f(x, y) e^{-2\pi i\left(\frac{(u - \frac{M}{2})x}{M} + \frac{(v - \frac{N}{2})y}{N}\right)} \\
                          & = \fourier(f)(u - \frac{M}{2}, v - \frac{N}{2})\qed
    \end{align*}
\end{task}

\begin{task}
    \begin{align*}
        (f * g) (x) & = \INT g(x') f(x - x')\, dx'                                                                                    \\
                    & = \INT g(x') \left(\INT \fourier(f)(\omega) e^{2\pi i \omega (x - x')}\, d\omega)\right)\, dx'                   \\
                    & = \INT g(x') \left(\INT \fourier(f)(\omega) e^{2\pi i \omega x}\cdot e^{-2\pi i \omega x}\, d\omega)\right)\, dx' \\
                    & = \INT \fourier(f)(\omega) e^{2\pi i \omega x} \left(\INT g(x') e^{-2\pi i \omega x}\, dx')\right)\, d\omega\\
                    & = \INT \fourier(f)(\omega)\fourier(g)(\omega) e^{2\pi i \omega x}\, d\omega\\
                    & = \fourier^{-1}(\fourier(F)\fourier(g))(x)\qed
    \end{align*}
\end{task}

\begin{task}
    \begin{align*}
        \fourier(\lambda f + \mu g)(u,v) & = \INT\INT (\lambda f(x, y) + \mu g(x, y))e^{-2\pi i(ux + vy)}\, dxdy \\
                                         & = \lambda \INT\INT f(x,y)e^{-2\pi i(ux + vy)}\, dxdy                  \\
                                         & \quad + \mu \INT\INT g(x,y)e^{-2\pi i(ux + vy)}\, dxdy                \\
                                         & = \lambda \fourier(f)(u,v) + \mu \fourier(g)(u,v) \qed
    \end{align*}
    \begin{align*}
        \fourier^{-1}(\lambda f + \mu g)(u,v) & = \INT\INT (\lambda f(x, y) + \mu g(x, y))e^{2\pi i(ux + vy)}\, dxdy \\
                                         & = \lambda \INT\INT f(x,y)e^{2\pi i(ux + vy)}\, dxdy                  \\
                                         & \quad + \mu \INT\INT g(x,y)e^{2\pi i(ux + vy)}\, dxdy                \\
                                         & = \lambda \fourier^{-1}(f)(u,v) + \mu \fourier^{-1}(g)(u,v) \qed
    \end{align*}
\end{task}

\begin{task}
    \begin{gather*}
        A_{\text{avg}}(x) = \frac{1}{n} \text{box}\left(\frac{x}{n}\right)\\
        \fourier(A_{\text{avg}})(\omega) = \frac{1}{n}\fourier\left(\text{box}\left(\frac{x}{n}\right)\right) = \text{sinc}(n\omega)
    \end{gather*}
    Die $\text{sinc}$-Funktion wird mit steigender Frequenz immer geringer. Hohe Frequenzen (Kanten) werden also gestaucht, wodurch es zu einer Glättung des Bildes kommt.
\end{task}
\end{document}