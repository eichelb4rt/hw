\documentclass[11pt]{article}
\usepackage{geometry}
\geometry{a4paper, top=20mm, left=10mm, right=10mm, bottom=20mm}
\usepackage{graphicx}
\usepackage{amsmath,amssymb,amsthm}
\usepackage{amssymb}
\usepackage[utf8]{inputenc}
\usepackage{fancyhdr}
\usepackage{lastpage}
\usepackage{enumerate}
\usepackage{enumitem}
\usepackage{multicol}
\usepackage{subcaption}
\usepackage{color}
\usepackage{scalerel}
%------------------------------------------ preamble
%----- fancyhdr
\fancyhead[L]{Name: Maurice Wenig}
\fancyhead[R]{Matrikelnummer: 178049}
\fancyfoot{}
\rfoot{Seite \thepage\ von \pageref{LastPage}}
\pagestyle{fancy}
%----- aufgaben
\newtheoremstyle{break}{}{5mm}{}{}{\bfseries}{}{0mm}
{\textbf{\thmname{#1}\thmnumber{ #2:} \thmnote{\textit{#3}}\newline}}
\theoremstyle{break}
\newtheorem{task}{Aufgabe}
%----- new commands
\newcommand{\Romannumeral}[1]{\MakeUppercase{\romannumeral #1}}
\newcommand{\set}[1]{\{#1\}}
\newcommand{\abs}[1]{\left\vert #1 \right\vert}
\newcommand{\norm}[1]{\left\| #1 \right\|}
\newcommand{\skal}[2]{\left\langle #1 | #2 \right\rangle}
\newcommand{\expected}[1]{\mathbb{E}\set{#1}}
\newcommand{\variance}[1]{\text{Var}\set{#1}}
%----- defs
\def\notiff{\mathrel{{\ooalign{\hidewidth$\not\phantom{"}$\hidewidth\cr$\iff$}}}}
\def\R{\ensuremath{\mathbb{R}}}
\def\1{\ensuremath{\mathbb{1}}}
\def\ndy{
    \textcolor{red} {\hfill not done yet!}
    \reversemarginpar
    \marginpar{\raggedleft\textcolor{red}{\rule{2mm}{2mm}}}
}
%----- homework number
\newcommand{\hw}{1}
%------------------------------------------ main
\begin{document}
%----- title
\begin{center}
\Large{Rechnersehen Theorieaufgaben}\\
\large{\hw. Übungsserie}
\end{center}
%----- tasks
\begin{task}
    Seitenlänge $H$ der quadratischen Fläche ist $\frac{7\,\text{mm}}{35\,\text{mm}} \cdot 500\,\text{mm} = 100\,\text{mm}$. Da wir $2^{10}$ Pixel an einer Seite haben, können wir $\frac{2^{9}\,\text{lp}}{100\,\text{mm}} = 5.12\,\frac{\text{lp}}{\text{mm}}$ auflösen.
\end{task}

\begin{task}
    $$\expected{\overline{g}(x, y)} = \expected{\frac{1}{K}\sum\limits_{i=1}^K f(x, y) + \eta_i(x, y)} = \expected{f(x, y)} + \frac{1}{K} \sum\limits_{i=1}^K \underbrace{\expected{\eta_i(x, y)}}_0 = f(x, y) \qed$$
    Da $\eta_i(x, y)$ für $i = 1,\dots,K$ paarweise unkorreliert sind, gilt:
    $$\sigma^2_{\overline{g}(x,y)} = \variance{\overline{g}(x,y) - \expected{\overline{g}(x,y)}} = \variance{\frac{1}{K} \sum\limits_{i=1}^K \eta_i(x, y)} = \frac{1}{K^2} \sum\limits_{i=1}^K \underbrace{\variance{\eta_i(x, y)}}_{\sigma^2_{\eta(x, y)}} = \frac{1}{K} \sigma^2_{\eta(x, y)} \qed$$
\end{task}

\begin{task}
    $$\underbrace{\zeta (0 + 0, 10 + 2, 20 + 1)}_{12} \neq \underbrace{\zeta (0, 10, 20)}_{10} + \underbrace{\zeta (0, 2, 1)}_1 \qed$$
\end{task}

\begin{task}
    Gegeben sei das Bild $f$, dann sieht die gewünschte Transformation $T$ wie folgt aus:
    $$T(f) = (L - 1)\cdot \frac{f - b}{a - b}$$
\end{task}
\end{document}