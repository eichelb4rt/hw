\documentclass[11pt]{article}
\usepackage{geometry}
\geometry{a4paper, top=20mm, left=10mm, right=10mm, bottom=20mm}
\usepackage{graphicx}
\usepackage{amsmath,amssymb,amsthm}
\usepackage{amssymb}
\usepackage[utf8]{inputenc}
\usepackage{fancyhdr}
\usepackage{lastpage}
\usepackage{enumerate}
\usepackage{enumitem}
\usepackage{multicol}
\usepackage{subcaption}
\usepackage{ifthen}
\usepackage{listings}
\usepackage{color}
\usepackage{scalerel}
%------------------------------------------ preamble
%----- fancyhdr
\fancyhead[R]{Übungsgruppe: 1 (Di 10-12)}
\fancyhead[C]{Name: Maurice Wenig}
\fancyhead[L]{Matrikelnummer: 178049}
\fancyfoot{}
\rfoot{Seite \thepage\ von \pageref{LastPage}}
\pagestyle{fancy}
%----- aufgaben
\newtheoremstyle{break}{}{5mm}{}{}{\bfseries}{}{0mm}
{\textbf{\thmname{#1}\thmnumber{ \hw.#2:} \thmnote{\textit{#3}}\newline}}
\theoremstyle{break}
\newtheorem{task}{Aufgabe}
%----- listings
\definecolor{mygreen}{rgb}{0,0.6,0}
\definecolor{mygray}{rgb}{0.5,0.5,0.5}
\definecolor{mymauve}{rgb}{0.58,0,0.82}
\lstset{ %
backgroundcolor=\color{white}, 
% choose the background color; you must add
basicstyle=\footnotesize, 
% the size of the fonts that are used for the code
breakatwhitespace=false, 
% sets if automatic breaks should only happen at whitespace
breaklines=true,
% sets automatic line breaking
captionpos=b,
% sets the caption-position to bottom
commentstyle=\color{mygreen},
% comment style
deletekeywords={...},
% if you want to delete keywords from the given language
extendedchars=true,
% lets you use non-ASCII characters; for 8-bits encodings only   
% adds a frame around the code
keepspaces=true, 
% keeps spaces in text, useful for keeping indentation of code
keywordstyle=\color{blue}, 
% keyword style
language=java, 
% the language of the code
otherkeywords={@Override},
% if you want to add more keywords to the set
numbers=left, 
% where to put the line-numbers; possible values(none, left, right)
numbersep=5pt, 
% how far the line-numbers are from the code
numberstyle=\tiny\color{mygray}, % the style that is used for the line-numbers
% if not set, frame-color can change on line-breaks to text-color
showspaces=false, 
% show spaces everywhere adding particular underscores;
showstringspaces=false, 
% underline spaces within strings only
showtabs=false, 
% show tabs within strings adding particular underscores
stepnumber=1, 
% the step between two line-numbers.
stringstyle=\color{mymauve}, 
% string literal style
tabsize=2,     
% sets default tabsize to 2 spaces
% show the filename of files included with \lstinputlisting;
emph={@Override},
emphstyle={\color{Goldenrod}}
}
%----- new commands
\newcommand{\Romannumeral}[1]{\MakeUppercase{\romannumeral #1}}
\newcommand{\set}[1]{\ensuremath{\{#1\}}}
\newcommand{\abs}[1]{\ensuremath{\left\vert #1 \right\vert}}
\newcommand{\norm}[1]{\ensuremath{\left\| #1 \right\|}}
\newcommand{\skal}[2]{\ensuremath{\left\langle #1 | #2 \right\rangle}}
\newcommand{\script}[1]{
    skripte/aufgabe#1.py
    \lstinputlisting{skripte/aufgabe#1.py}
}
%----- defs
\def\notiff{\mathrel{{\ooalign{\hidewidth$\not\phantom{"}$\hidewidth\cr$\iff$}}}}
\def\R{\ensuremath{\mathbb{R}}}
\def\1{\ensuremath{{\normalfont\hbox{1\kern-0.18em \vrule width .6pt}}}}
\def\ndy{
    \textcolor{red} {\hfill not done yet!}
    \reversemarginpar
    \marginpar{\raggedleft\textcolor{red}{\rule{2mm}{2mm}}}
}
%----- homework number
\newcommand{\hw}{5}
%------------------------------------------ main
\begin{document}
%----- title
\begin{center}
\Large{Numerische Mathematik}\\
\large{\hw. Übungsserie}
\end{center}
%----- tasks
\begin{task}
    \hfill\vspace{-5mm}
    \begin{enumerate}[label={(\alph*)}]
        \item \hfill\vspace{-5mm}
        \begin{align*}
            (L;b) & = \left(\begin{array}{cc|c}
                1 & 0 & 1\\
                \frac{104}{105} & 1 & 2
            \end{array}\right)\rightarrow\left(\begin{array}{cc|c}
                1 & 0 & 1\\
                0 & 1 & 1
            \end{array}\right)\\
            \implies \overline{z} &= \begin{pmatrix}
                1\\
                1.01
            \end{pmatrix}\\
            (R,\overline{z}) &= \left(\begin{array}{cc|c}
                1.05 & 1.02 & 1\\
                1.04 & 1.02 & 1.01
            \end{array}\right)\rightarrow\left(\begin{array}{cc|c}
                1.05 & 1.02 & 1\\
                0 & 9.71\cdot 10^{-3} & 1.95\cdot 10^{-2}
            \end{array}\right)\\
            \implies \overline{x} &= \begin{pmatrix}
                -9.90\cdot 10^{-1}\\
                2
            \end{pmatrix}
        \end{align*}
        \item \begin{align*}
            \Delta_3 &= \begin{pmatrix}
                0\\
                0.99
            \end{pmatrix}\\
            \Delta_6 &= \begin{pmatrix}
                5\cdot 10^{-4}\\
                9.896\cdot 10^{-1}
            \end{pmatrix}
        \end{align*}
        \item \hfill\vspace{-5mm}
        \begin{align*}
            (A;\Delta_6) &= \left(\begin{array}{cc|c}
                1.05 & 1.02 & 5\cdot 10^{-4}\\
                1.04 & 1.02 & 9.896\cdot 10^{-1}
            \end{array}\right) \rightarrow \left(\begin{array}{cc|c}
                1.05 & 1.02 & 5\cdot 10^{-4}\\
                0 & 9.71\cdot 10^{-3} & 9.90\cdot 10^{-1}
            \end{array}\right)\\
            \overline{y} &= \begin{pmatrix}
                -99.1\\
                102
            \end{pmatrix}
        \end{align*}
    \end{enumerate}
\end{task}

\begin{task}
    \hfill\vspace{-5mm}
    \begin{enumerate}[label={(\alph*)}]
        \item IA: $x^{(0)}$ ist Wahrscheinlichkeitsvektor\\
        IV: $x^{(k)}$ ist Wahrscheinlichkeitsvektor\\
        IS: 
        \begin{align*}
            \forall_{i=1, \ldots, n}: \left(Wx^{(k)}\right)_i &= \sum\limits_{j=1}^{n} \underbrace{W_{i,j}}_{\geq 0} \underbrace{x^{(k)}_j}_{\geq 0} \geq 0\\
            \sum\limits_{i=1}^n \left(Wx^{(k)}\right)_i &= \sum\limits_{i=1}^n \sum\limits_{j=1}^{n} W_{i,j} x^{(k)}_j = \sum\limits_{j=1}^n \sum\limits_{i=1}^{n} W_{i,j} x^{(k)}_j = \sum\limits_{i=1}^n x^{(k)}_j = 1
        \end{align*}
        $\implies x^{(k+1)}$ ist Wahrscheinlichkeitsvektor $\hfill\qed$

        \item $\forall_{i=1, \ldots, n}: y_i = 1 \implies \forall_{i=1, \ldots, n}: ((W^T - \1)y)_i = -1 + \sum\limits_{j=1}^{n} W^T_{i,j} y_j = -1 + \sum\limits_{j=1}^{n} W^T_{i,j} = 0$\\
        $\implies (W^T - \1)y = 0 \iff W^Ty = y \iff y$ ist Eigenvektor von $W^T$ mit Eigenwert $1$\\
        $\implies 1$ ist Eigenwert von $W$ $\hfill\qed$

        \item $W x^{(\infty)} = W \lim\limits_{k\rightarrow\infty} x^{(k)} = \lim\limits_{k\rightarrow\infty} W x^{(k)} = \lim\limits_{k\rightarrow\infty} x^{(k+1)} = x^{(\infty)}\hfill\qed$
        \item \hfill\vspace{-5mm}
        \begin{align*}
            x^{(5)} &\approx \begin{pmatrix}
                0.29884000\\
                0.38248000\\
                0.31868000
            \end{pmatrix}\\
            x^{(10)} &\approx \begin{pmatrix}
                0.29824163\\
                0.38598741\\
                0.31577096
            \end{pmatrix}\\
            x^{(15)} &\approx \begin{pmatrix}
                0.29824564\\
                0.38596477\\
                0.31578959
            \end{pmatrix}
        \end{align*}
    \end{enumerate}
\end{task}
\end{document}