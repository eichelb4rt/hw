\documentclass[11pt]{article}
\usepackage{geometry}
\geometry{a4paper, top=20mm, left=10mm, right=10mm, bottom=20mm}
\usepackage{graphicx}
\usepackage{amsmath,amssymb,amsthm}
\usepackage{amssymb}
\usepackage[utf8]{inputenc}
\usepackage{fancyhdr}
\usepackage{lastpage}
\usepackage{enumerate}
\usepackage{enumitem}
\usepackage{multicol}
\usepackage{subcaption}
\usepackage{ifthen}
%------------------------------------------ preamble
%----- fancyhdr
\fancyhead[R]{Übungsgruppe: 1 (Di 10-12)}
\fancyhead[C]{Name: Maurice Wenig}
\fancyhead[L]{Matrikelnummer: 178049}
\fancyfoot{}
\rfoot{Seite \thepage\ von \pageref{LastPage}}
\pagestyle{fancy}
%----- aufgaben
\newtheoremstyle{break}{}{5mm}{}{}{\bfseries}{}{0mm}
{\textbf{\thmname{#1}\thmnumber{ #2:} \thmnote{\textit{#3}}\newline}}
\theoremstyle{break}
\newtheorem{task}{Aufgabe}
%----- new commands
\newcommand{\Romannumeral}[1]{\MakeUppercase{\romannumeral #1}}
\newcommand{\notiff}{\mathrel{{\ooalign{\hidewidth$\not\phantom{"}$\hidewidth\cr$\iff$}}}}
\newcommand{\set}[1]{\ensuremath{\{#1\}}}
\newcommand{\abs}[1]{\ensuremath{\left\vert #1 \right\vert}}
%------------------------------------------ main
\begin{document}
%----- title
\begin{center}
\Large{Numerische Mathematik}\\
\large{1. Übungsserie}
\end{center}
%----- tasks
\begin{task}
\begin{flalign*}
    0\; 00 \; 00 \;\;\; \text{ entspricht: } (-1)^1 \cdot \frac{0}{2^2} \cdot 2^{1-1} &= 0&\\
    0\; 00 \; 01 \;\;\; \text{ entspricht: } (-1)^1 \cdot \frac{1}{2^2} \cdot 2^{1-1} &= 0.25&\\
    0\; 00 \; 10 \;\;\; \text{ entspricht: } (-1)^1 \cdot \frac{2}{2^2} \cdot 2^{1-1} &= 0.5&\\
    0\; 00 \; 11 \;\;\; \text{ entspricht: } (-1)^1 \cdot \frac{3}{2^2} \cdot 2^{1-1} &= 0.75&\\
    0\; 01 \; 00 \;\;\; \text{ entspricht: } (-1)^1 \cdot (1 + \frac{0}{2^2}) \cdot 2^{1-1} &= 1&\\
    0\; 01 \; 01 \;\;\; \text{ entspricht: } (-1)^1 \cdot (1 + \frac{1}{2^2}) \cdot 2^{1-1} &= 1.25&\\
    0\; 01 \; 10 \;\;\; \text{ entspricht: } (-1)^1 \cdot (1 + \frac{2}{2^2}) \cdot 2^{1-1} &= 1.5&\\
    0\; 01 \; 11 \;\;\; \text{ entspricht: } (-1)^1 \cdot (1 + \frac{3}{2^2}) \cdot 2^{1-1} &= 1.75&\\
    0\; 10 \; 00 \;\;\; \text{ entspricht: } (-1)^1 \cdot (1 + \frac{0}{2^2}) \cdot 2^{2-1} &= 2&\\
    0\; 10 \; 01 \;\;\; \text{ entspricht: } (-1)^1 \cdot (1 + \frac{1}{2^2}) \cdot 2^{2-1} &= 2.5&\\
    0\; 10 \; 10 \;\;\; \text{ entspricht: } (-1)^1 \cdot (1 + \frac{2}{2^2}) \cdot 2^{2-1} &= 3&\\
    0\; 10 \; 11 \;\;\; \text{ entspricht: } (-1)^1 \cdot (1 + \frac{3}{2^2}) \cdot 2^{2-1} &= 3.5&
\end{flalign*}
\end{task}
\end{document}
