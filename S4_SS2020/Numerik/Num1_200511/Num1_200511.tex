\documentclass[11pt]{article}
\usepackage{geometry}
\geometry{a4paper, top=20mm, left=10mm, right=10mm, bottom=20mm}
\usepackage{graphicx}
\usepackage{amsmath,amssymb,amsthm}
\usepackage{amssymb}
\usepackage[utf8]{inputenc}
\usepackage{fancyhdr}
\usepackage{lastpage}
\usepackage{enumerate}
\usepackage{enumitem}
\usepackage{multicol}
\usepackage{subcaption}
\usepackage{ifthen}
\usepackage{listings}
\usepackage{color}
\usepackage{scalerel}
%------------------------------------------ preamble
%----- fancyhdr
\fancyhead[R]{Übungsgruppe: 1 (Di 10-12)}
\fancyhead[C]{Name: Maurice Wenig}
\fancyhead[L]{Matrikelnummer: 178049}
\fancyfoot{}
\rfoot{Seite \thepage\ von \pageref{LastPage}}
\pagestyle{fancy}
%----- aufgaben
\newtheoremstyle{break}{}{5mm}{}{}{\bfseries}{}{0mm}
{\textbf{\thmname{#1}\thmnumber{ #2:} \thmnote{\textit{#3}}\newline}}
\theoremstyle{break}
\newtheorem{task}{Aufgabe}
%----- listings
\definecolor{mygray}{rgb}{0.5,0.5,0.5}
\lstset{
    language=python,
    basicstyle=\ttfamily,
    numbers=left,
    numberstyle=\tiny\color{mygray},
    breaklines=true
}
%----- new commands
\newcommand{\Romannumeral}[1]{\MakeUppercase{\romannumeral #1}}
\newcommand{\notiff}{\mathrel{{\ooalign{\hidewidth$\not\phantom{"}$\hidewidth\cr$\iff$}}}}
\newcommand{\set}[1]{\ensuremath{\{#1\}}}
\newcommand{\abs}[1]{\ensuremath{\left\vert #1 \right\vert}}
\newcommand{\norm}[1]{\ensuremath{\left\| #1 \right\|}}
\newcommand{\skal}[2]{\ensuremath{\left\langle #1 | #2 \right\rangle}}
\newcommand{\R}{\ensuremath{\mathbb{R}}}
\newcommand{\script}[1]{
    skripte/aufgabe#1.sh
    \lstinputlisting{skripte/aufgabe#1.py}
}
\newcommand{\ndy}{
    \textcolor{red} {\hfill not done yet!}
    \reversemarginpar
    \marginpar{\raggedleft\textcolor{red}{\rule{2mm}{2mm}}}
}
%------------------------------------------ main
\begin{document}
%----- title
\begin{center}
\Large{Numerische Mathematik}\\
\large{1. Übungsserie}
\end{center}
%----- tasks
\begin{task}
\begin{flalign*}
    &1\; 00 \; 00 \;\;\; \text{ entspricht: } (-1)^0 \cdot \frac{0}{2^2} \cdot 2^{1-1} &&\hspace{-5cm}= 0&\\
    &0\; 00 \; 00 \;\;\; \text{ entspricht: } (-1)^0 \cdot \frac{0}{2^2} \cdot 2^{1-1} &&\hspace{-5cm}= 0&\\
    &0\; 00 \; 01 \;\;\; \text{ entspricht: } (-1)^0 \cdot \frac{1}{2^2} \cdot 2^{1-1} &&\hspace{-5cm}= 0.25&\\
    &0\; 00 \; 10 \;\;\; \text{ entspricht: } (-1)^0 \cdot \frac{2}{2^2} \cdot 2^{1-1} &&\hspace{-5cm}= 0.5&\\
    &0\; 00 \; 11 \;\;\; \text{ entspricht: } (-1)^0 \cdot \frac{3}{2^2} \cdot 2^{1-1} &&\hspace{-5cm}= 0.75&\\
    &0\; 01 \; 00 \;\;\; \text{ entspricht: } (-1)^0 \cdot (1 + \frac{0}{2^2}) \cdot 2^{1-1} &&\hspace{-5cm}= 1&\\
    &0\; 01 \; 01 \;\;\; \text{ entspricht: } (-1)^0 \cdot (1 + \frac{1}{2^2}) \cdot 2^{1-1} &&\hspace{-5cm}= 1.25&\\
    &0\; 01 \; 10 \;\;\; \text{ entspricht: } (-1)^0 \cdot (1 + \frac{2}{2^2}) \cdot 2^{1-1} &&\hspace{-5cm}= 1.5&\\
    &0\; 01 \; 11 \;\;\; \text{ entspricht: } (-1)^0 \cdot (1 + \frac{3}{2^2}) \cdot 2^{1-1} &&\hspace{-5cm}= 1.75&\\
    &0\; 10 \; 00 \;\;\; \text{ entspricht: } (-1)^0 \cdot (1 + \frac{0}{2^2}) \cdot 2^{2-1} &&\hspace{-5cm}= 2&\\
    &0\; 10 \; 01 \;\;\; \text{ entspricht: } (-1)^0 \cdot (1 + \frac{1}{2^2}) \cdot 2^{2-1} &&\hspace{-5cm}= 2.5&\\
    &0\; 10 \; 10 \;\;\; \text{ entspricht: } (-1)^0 \cdot (1 + \frac{2}{2^2}) \cdot 2^{2-1} &&\hspace{-5cm}= 3&\\
    &0\; 10 \; 11 \;\;\; \text{ entspricht: } (-1)^0 \cdot (1 + \frac{3}{2^2}) \cdot 2^{2-1} &&\hspace{-5cm}= 3.5&
\end{flalign*}
\end{task}

\begin{task}
    \ndy
\end{task}

\begin{task}
    \hfill\vspace{-5mm}
    \begin{enumerate} [label={(\alph*)}]
        \item Sei $S\in \R^n$ eine Orthonormalbasis aus Eigenvektoren von $A^TA$, sodass $S^TA^TAS=D$ eine Diagonalmatrix aus Eigenwerten $\lambda_1, \dots, \lambda_n$ von $A^TA$ ist (Spektralsatz). Weiterhin sei $x = Sy,\ x,y\in\R^n$.
        $$\norm{A}_2^2 = (\max\limits_{\norm{x}_2 = 1} \norm{Ax}_2)^2 = \max\limits_{\norm{Sy}_2 = 1} \skal{ASy}{ASy} = \max\limits_{\norm{y}_2 = 1} \skal{S^TA^TASy}{y} = \max\limits_{\norm{y}_2 = 1} \sum\limits_{i=1}^n \lambda_i y_i^2$$
        Diese Summe ist maximal mit $y=e_i=:e_{max}$, mit $i\in \set{1,\dots, n}$, sodass $\lambda_i = \lambda_{max} := \max\limits_{i\in\set{1,\dots,n}} \lambda_i$:
        \begin{align*}
            \sum\limits_{i=1}^n \lambda_i y_i^2&\leq \sum\limits_{i=1}^n \lambda_{max} y_i^2 = \lambda_{max} \norm{y}_2^2 = \lambda_{max}\\
            \sum\limits_{i=1}^n \lambda_i e_{max_i}^2 &= \lambda_{max}\\
            \implies \norm{A}_2 &= \underline{\underline{\sqrt{\lambda_{max}}}}
        \end{align*}
        \item \begin{align*}
            \norm{A}_\infty &= \max\limits_{x\in \R^n, \norm{x}_\infty = 1}\norm{Ax}_\infty\\
            &= \max\limits_{x\in \R^n, \norm{x}_\infty = 1}\norm{\sum\limits_{i=1}^n\left(
            \begin{matrix}
                A_{1,j}x_j\\
                A_{2,j}x_j\\
                \vdots\\
                A_{m,j}x_j
            \end{matrix}
            \right)}_\infty\\
            &= \max\limits_{x\in \R^n, \norm{x}_\infty = 1}\left(\max\limits_{i\in\set{1,\dots,m}}\abs{\sum\limits_{j=1}^n A_{i,j}x_j}\right)\\
            &= \max\limits_{i\in\set{1,\dots,m}}\left(\max\limits_{x\in \R^n, \norm{x}_\infty = 1}\abs{\sum\limits_{j=1}^n A_{i,j}x_j}\right)\\
            &\stackrel{\mbox{\tiny (1)}}{=} \underline{\underline{\max\limits_{i\in\set{1,\dots,m}}\sum\limits_{j=1}^n \abs{A_{i,j}}}}
        \end{align*}
        $(1)\ \sum\limits_{j=1}^n A_{i,j}x_j$ ist maximal mit $x_j = \frac{\abs{A_{i,j}}}{A_{i,j}}$, da $\forall j\in\set{1,\dots,n}:\abs{x_j}\leq 1$
    \end{enumerate}
\end{task}

\begin{task}
    \hfill\vspace{-5mm}
    \begin{enumerate} [label={(\alph*)}]
        \item Sei $L>0$ die Lipschitz-Konstante. $$\forall x\in [a,b]\ \forall \epsilon>0\ \exists \delta(\epsilon, x)=\frac{\epsilon}{L}\ \forall y\in [a,b]: \norm{f(x)-f(y)} \geq \epsilon \implies \norm{x-y}\cdot L \geq \epsilon \implies \norm{x-y}\geq\delta(\epsilon, x)$$
        $\implies$ $f$ ist stetig in $[a,b]$.\vspace{3mm}\\
        Falls $L=0$, dann $\forall x,y\in [a,b]: \norm{f(x)-f(y)}=0$\\
        $\implies$ $f$ ist stetig in $[a,b]$.
    \end{enumerate}
\end{task}
\end{document}