\documentclass[11pt]{article}
\usepackage{geometry}
\geometry{a4paper, top=20mm, left=10mm, right=10mm, bottom=20mm}
\usepackage{graphicx}
\usepackage{amsmath,amssymb,amsthm}
\usepackage{amssymb}
\usepackage[utf8]{inputenc}
\usepackage{fancyhdr}
\usepackage{lastpage}
\usepackage{enumerate}
\usepackage{enumitem}
\usepackage{multicol}
\usepackage{subcaption}
\usepackage{ifthen}
\usepackage{listings}
\usepackage{color}
\usepackage{scalerel}
%------------------------------------------ preamble
%----- fancyhdr
\fancyhead[R]{Übungsgruppe: 1 (Di 10-12)}
\fancyhead[C]{Name: Maurice Wenig}
\fancyhead[L]{Matrikelnummer: 178049}
\fancyfoot{}
\rfoot{Seite \thepage\ von \pageref{LastPage}}
\pagestyle{fancy}
%----- aufgaben
\newtheoremstyle{break}{}{5mm}{}{}{\bfseries}{}{0mm}
{\textbf{\thmname{#1}\thmnumber{ \hw.#2:} \thmnote{\textit{#3}}\newline}}
\theoremstyle{break}
\newtheorem{task}{Aufgabe}
%----- listings
\definecolor{mygreen}{rgb}{0,0.6,0}
\definecolor{mygray}{rgb}{0.5,0.5,0.5}
\definecolor{mymauve}{rgb}{0.58,0,0.82}
\lstset{ %
backgroundcolor=\color{white}, 
% choose the background color; you must add
basicstyle=\footnotesize, 
% the size of the fonts that are used for the code
breakatwhitespace=false, 
% sets if automatic breaks should only happen at whitespace
breaklines=true,
% sets automatic line breaking
captionpos=b,
% sets the caption-position to bottom
commentstyle=\color{mygreen},
% comment style
deletekeywords={...},
% if you want to delete keywords from the given language
extendedchars=true,
% lets you use non-ASCII characters; for 8-bits encodings only   
% adds a frame around the code
keepspaces=true, 
% keeps spaces in text, useful for keeping indentation of code
keywordstyle=\color{blue}, 
% keyword style
language=java, 
% the language of the code
otherkeywords={@Override},
% if you want to add more keywords to the set
numbers=left, 
% where to put the line-numbers; possible values(none, left, right)
numbersep=5pt, 
% how far the line-numbers are from the code
numberstyle=\tiny\color{mygray}, % the style that is used for the line-numbers
% if not set, frame-color can change on line-breaks to text-color
showspaces=false, 
% show spaces everywhere adding particular underscores;
showstringspaces=false, 
% underline spaces within strings only
showtabs=false, 
% show tabs within strings adding particular underscores
stepnumber=1, 
% the step between two line-numbers.
stringstyle=\color{mymauve}, 
% string literal style
tabsize=2,     
% sets default tabsize to 2 spaces
% show the filename of files included with \lstinputlisting;
emph={@Override},
emphstyle={\color{Goldenrod}}
}
%----- new commands
\newcommand{\Romannumeral}[1]{\MakeUppercase{\romannumeral #1}}
\newcommand{\notiff}{\mathrel{{\ooalign{\hidewidth$\not\phantom{"}$\hidewidth\cr$\iff$}}}}
\newcommand{\set}[1]{\ensuremath{\{#1\}}}
\newcommand{\abs}[1]{\ensuremath{\left\vert #1 \right\vert}}
\newcommand{\norm}[1]{\ensuremath{\left\| #1 \right\|}}
\newcommand{\skal}[2]{\ensuremath{\left\langle #1 | #2 \right\rangle}}
\newcommand{\R}{\ensuremath{\mathbb{R}}}
\newcommand{\komp}{\ensuremath{\underline{\kappa}}}
\newcommand{\result}[1]{\ensuremath{\underline{\underline{#1}}}}
\newcommand{\script}[1]{
    skripte/aufgabe#1.py
    \lstinputlisting{skripte/aufgabe#1.py}
}
\newcommand{\ndy}{
    \textcolor{red} {\hfill not done yet!}
    \reversemarginpar
    \marginpar{\raggedleft\textcolor{red}{\rule{2mm}{2mm}}}
}
%..... homework number
\newcommand{\hw}{2}
%------------------------------------------ main
\begin{document}
%----- title
\begin{center}
\Large{Numerische Mathematik}\\
\large{\hw. Übungsserie}
\end{center}
%----- tasks
\begin{task}
    \hfill\vspace{-5mm}
    \begin{enumerate}[label={(\alph*)}]
        \item $\komp^{rel}(f,x) = \max_{i,j} \frac{\abs{x_j}}{\abs{f_i(x)}}\cdot\abs{\frac{\partial f_i}{\partial x_j}(x)} = \frac{x}{\sqrt{x}}\cdot \frac{1}{2} x^{-\frac{1}{2}}=\result{\frac{1}{2}}$
        \item $\komp^{rel}(f,x) = \max_{i,j} \frac{\abs{x_j}}{\abs{f_i(x)}}\cdot\abs{\frac{\partial f_i}{\partial x_j}(x)}$,
        \begin{align*}
            \frac{\abs{x_1}}{\abs{f(x)}}\cdot\abs{\frac{\partial f}{\partial x_1}(x)} &= \frac{\abs{x_1}}{\abs{x_1^{x_2}}}\cdot \abs{ x_2 x_1^{x_2 -1}}=\abs{x_2}\\
            \frac{\abs{x_2}}{\abs{f(x)}}\cdot\abs{\frac{\partial f}{\partial x_2}(x)} &= \frac{\abs{x_2}}{\abs{x_1^{x_2}}}\cdot \abs{ x_1^{x_2}\ln x_1}=\abs{x_2}\cdot\abs{\ln x_1}
        \end{align*}
        $\implies \komp^{rel}(f,x) = \begin{cases}
            x_2 & \text{falls } e^{-1} \leq x \leq e\\
            x_2\ln x_1 & \text{sonst}
        \end{cases}$
    \end{enumerate}
\end{task}
\begin{task}
    \hfill\vspace{-5mm}
    \begin{enumerate}[label={(\alph*)}]
        \item \ndy
        \item \ndy
        \item \hfill\vspace{-5mm}
        \begin{align*}
            \overline{u} &= 4.000 \times 10^0       &\overline{v}&=3.990\times 10^0                                                 & \overline{w}&=1.997 \times 10^0\\
            \epsilon &=0                            & \epsilon &=0                                                                  & \epsilon &\approx 2.5\times 10^{-4}\\\\
            \overline{y}_2 &= -3.997 \times 10^0    &\overline{y}_1 = \frac{\overline{p}}{2} + \overline{w} &= -3.000\times 10^{-3} & \overline{y}_1 = \frac{\overline{q}}{\overline{y}_2} &= 2,501\times 10^{-3}\\
            \epsilon &\approx 1,2\times 10^{-4}     &\epsilon &\approx 2\times 10^{-1}                                              & \epsilon &\approx 2.2\times 10^{-4}
        \end{align*}
    \end{enumerate}
\end{task}

\begin{task}
    \ndy
\end{task}

\begin{task}
    Da $\abs{f(x)}\leq 1$ ist der absolute Rundungsfehler $\abs{\delta} \leq \epsilon$.
    Der Rundungsfehler von $W(h)$ ist höchstens $\frac{\epsilon}{h}+o(\delta)$
    \begin{align*}
        f(a + h) &= f(a) + f'(a) h + \frac{f''(a)}{2} h^2 + \frac{f'''(\xi_1)}{6}h^3\\
        f(a - h) &= f(a) - f'(a) h + \frac{f''(a)}{2} h^2 - \frac{f'''(\xi_2)}{6}h^3
    \end{align*}
    mit $\xi_1$ zwischen $a$ und $a+h$, $\xi_2$ zwischen $a$ und $a-h$
        $$W(h) = \frac{f(a+h) - f(a-h)}{2h} = \frac{2f'(a) h + \frac{f'''(\xi_1) + f'''(\xi_2)}{6}h^3}{2h} = f'(a) + \frac{f'''(\xi_1) + f'''(\xi_2)}{12}h^2$$
    Dadurch ist der Verfahrensfehler höchstens $\frac{h^2}{6}$, denn $f'''(\xi_1) + f'''(\xi_2)\leq 2$.
    Der kleinste Fehler ist somit bei $\frac{\epsilon}{h}=\frac{h^2}{6}$ zu erwarten. 
    $\implies h=(6\epsilon)^\frac{1}{3} = \result{3^\frac{1}{3}\cdot 2^{-\frac{52}{3}}}$\vspace{3mm}\\
    Nun zum Test: \\
    \begin{center}
        \begin{tabular}{r|l}
            $h$ & $f(1) - W(h)$\\\hline
            $2^{-10}$ & $8.587876854\times 10^{-8}$\\
            $2^{-13}$ & $1.341690758\times 10^{-9}$\\
            $2^{-15}$ & $8.385958594\times 10^{-11}$\\
            $2^{-16}$ & $2.201394622\times 10^{-11}$\\
            $3^\frac{1}{3}\cdot 2^{-\frac{52}{3}}$ & $1.860733789\times 10^{-13}$\\
            $2^{-17}$ & $1.860733789\times 10^{-13}$\\
            $2^{-18}$ & $-7.089884235\times 10^{-12}$\\
            $2^{-20}$ & $-2.164179946\times 10^{-11}$\\
            $2^{-23}$ & $-7.984946038\times 10^{-11}$
        \end{tabular}
    \end{center}
    Unser vorhergesagtes $h$ funktioniert super und teilt sich unter den gewählten Werten mit $2^{17}$ den Thron.
\end{task}
\end{document}