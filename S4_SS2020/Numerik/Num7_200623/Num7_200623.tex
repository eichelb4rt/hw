\documentclass[11pt]{article}
\usepackage{geometry}
\geometry{a4paper, top=20mm, left=10mm, right=10mm, bottom=20mm}
\usepackage{graphicx}
\usepackage{amsmath,amssymb,amsthm}
\usepackage{amssymb}
\usepackage[utf8]{inputenc}
\usepackage{fancyhdr}
\usepackage{lastpage}
\usepackage{enumerate}
\usepackage{enumitem}
\usepackage{multicol}
\usepackage{subcaption}
\usepackage{ifthen}
\usepackage{listings}
\usepackage{color}
\usepackage{scalerel}
%------------------------------------------ preamble
%----- fancyhdr
\fancyhead[R]{Übungsgruppe: 1 (Di 10-12)}
\fancyhead[C]{Name: Maurice Wenig}
\fancyhead[L]{Matrikelnummer: 178049}
\fancyfoot{}
\rfoot{Seite \thepage\ von \pageref{LastPage}}
\pagestyle{fancy}
%----- aufgaben
\newtheoremstyle{break}{}{5mm}{}{}{\bfseries}{}{0mm}
{\textbf{\thmname{#1}\thmnumber{ \hw.#2:} \thmnote{\textit{#3}}\newline}}
\theoremstyle{break}
\newtheorem{task}{Aufgabe}
%----- listings
\definecolor{mygreen}{rgb}{0,0.6,0}
\definecolor{mygray}{rgb}{0.5,0.5,0.5}
\definecolor{mymauve}{rgb}{0.58,0,0.82}
\lstset{ %
backgroundcolor=\color{white}, 
% choose the background color; you must add
basicstyle=\footnotesize, 
% the size of the fonts that are used for the code
breakatwhitespace=false, 
% sets if automatic breaks should only happen at whitespace
breaklines=true,
% sets automatic line breaking
captionpos=b,
% sets the caption-position to bottom
commentstyle=\color{mygreen},
% comment style
deletekeywords={...},
% if you want to delete keywords from the given language
extendedchars=true,
% lets you use non-ASCII characters; for 8-bits encodings only   
% adds a frame around the code
keepspaces=true, 
% keeps spaces in text, useful for keeping indentation of code
keywordstyle=\color{blue}, 
% keyword style
language=java, 
% the language of the code
otherkeywords={@Override},
% if you want to add more keywords to the set
numbers=left, 
% where to put the line-numbers; possible values(none, left, right)
numbersep=5pt, 
% how far the line-numbers are from the code
numberstyle=\tiny\color{mygray}, % the style that is used for the line-numbers
% if not set, frame-color can change on line-breaks to text-color
showspaces=false, 
% show spaces everywhere adding particular underscores;
showstringspaces=false, 
% underline spaces within strings only
showtabs=false, 
% show tabs within strings adding particular underscores
stepnumber=1, 
% the step between two line-numbers.
stringstyle=\color{mymauve}, 
% string literal style
tabsize=2,     
% sets default tabsize to 2 spaces
% show the filename of files included with \lstinputlisting;
emph={@Override},
emphstyle={\color{Goldenrod}}
}
%----- new commands
\newcommand{\Romannumeral}[1]{\MakeUppercase{\romannumeral #1}}
\newcommand{\set}[1]{\ensuremath{\{#1\}}}
\newcommand{\abs}[1]{\ensuremath{\left\vert #1 \right\vert}}
\newcommand{\norm}[1]{\ensuremath{\left\| #1 \right\|}}
\newcommand{\skal}[2]{\ensuremath{\left\langle #1 | #2 \right\rangle}}
\newcommand{\script}[1]{
    skripte/aufgabe#1.py
    \lstinputlisting{skripte/aufgabe#1.py}
}
%----- defs
\def\notiff{\mathrel{{\ooalign{\hidewidth$\not\phantom{"}$\hidewidth\cr$\iff$}}}}
\def\R{\ensuremath{\mathbb{R}}}
\def\1{\ensuremath{\mathbb{1}}}
\def\ndy{
    \textcolor{red} {\hfill not done yet!}
    \reversemarginpar
    \marginpar{\raggedleft\textcolor{red}{\rule{2mm}{2mm}}}
}
%----- homework number
\newcommand{\hw}{7}
%------------------------------------------ main
\begin{document}
%----- title
\begin{center}
\Large{Numerische Mathematik}\\
\large{\hw. Übungsserie}
\end{center}
%----- tasks
\begin{task}
    \hfill\vspace{-5mm}
    \begin{enumerate}[label={(\alph*)}]
        \item \begin{align*}
            v &\approx \begin{pmatrix}
                2.92\\
                1.07\\
                1.07
            \end{pmatrix}& H_v &\approx \begin{pmatrix}
                0.580 & 0.577 & 0.577\\
                0.577 & 0.789 & -0.211\\
                0.577 & -0.211 & 0.789
            \end{pmatrix}& H_v\cdot A &\approx\begin{pmatrix}
                1.85 & -1.94\\
                0 & -0.00165\\
                0 & -0.0383
            \end{pmatrix}\\
            w &\approx\begin{pmatrix}
                -0.04\\
                -0.0383
            \end{pmatrix}& H_w &\approx \begin{pmatrix}
                -0.04 & -1\\
                -1 & 0.04
            \end{pmatrix}& H_w\cdot A' &\approx \begin{pmatrix}
                0.0383\\
                0
            \end{pmatrix}\\
            &&Q = H_v\cdot\begin{pmatrix}
                1 & 0\\
                0 & H_w
            \end{pmatrix} &\approx \begin{pmatrix}
                -0.58 & 0.554 & 0.554\\
                -0.577 & 0.243 & -0.797\\
                -0.577 & -0.797 & 0.243
            \end{pmatrix}\\
            &&R&\approx \begin{pmatrix}
                1.85 & -1.94\\
                0 & 0.0383\\
                0 & 0
            \end{pmatrix}\\
        \end{align*}
        Anmerkung: Ich habe vergessen, bei den Normen / der Berechnung von $\alpha$ zu runden. Das habe ich allerdings zu spät bermerkt und jetzt will ich nicht alles nochmal umschreiben. Außerdem haben wir am Anfang mal definiert, dass am Ende einer Gleitkommazahl nicht nur 9en sind. Deswegen wurde die ein oder andere $0.0399$ zu einer $0.04$ etc.
        \item \begin{align*}
            Q^T b &\approx \begin{pmatrix}
                -0.003\\
                0.311\\
                1.35
            \end{pmatrix}& R_1 &\approx \begin{pmatrix}
                1.85 & -1.94\\
                0 & 0.0383
            \end{pmatrix}& (R_1;b_1) &\approx \left(\begin{array}{cc|c}
                1.85 & -1.94 & -0.003\\
                0 & 0.0383 & 0.311
            \end{array}\right)\\
            &&x&\approx\begin{pmatrix}
                8.5\\
                8.12
            \end{pmatrix}
        \end{align*}
    \end{enumerate}
\end{task}
\begin{task}
    \ndy
\end{task}
\begin{task}
    \hfill\vspace{-5mm}
    \begin{enumerate}[label={(\alph*)}]
        \item \begin{align*}
            \sum\limits_{i,j=1}^{n} (B_{i,j})^2 &=\sum\limits_{i,j=1}^{n} \left((Q^T A Q)_{i,j}\right)^2\\
            &=\sum\limits_{i,j=1}^{n} \left(\sum\limits_{l,k=1}^{n} Q^T_{i,l}\cdot A_{l,k}\cdot Q_{k,j}\right)^2\\
            &=\sum\limits_{i,j=1}^{n}\left(\sum\limits_{l_1,l_2,k_1,k_2=1}^{n} Q^T_{i,l_1}\cdot A_{l_1,k_1}\cdot Q_{k_1,j}\cdot Q^T_{i,l_2}\cdot A_{l_2,k_2}\cdot Q_{k_2,j}\right)\\
            &=\sum\limits_{l_1,l_2,k_1,k_2=1}^{n} A_{l_1,k_1}\cdot A_{l_2,k_2}\cdot \underbrace{\sum\limits_{i,j=1}^{n} Q_{l_1,i}\cdot Q^T_{i,l_2}\cdot Q_{k_1,j}\cdot Q^T_{j,k_2}}_{1\text{ falls } l_1=l_2\text{ und } k_1=k_2,\ 0\text{ sonst}}\\
            &=\sum\limits_{i,j=1}^{n} (A_{i,j})^2\qed
        \end{align*}
        \item Nach der Hauptachsentransformation gibt es für jede symmetrische Matrix $A\in M_n(\R)$ eine diagonalisierende Matrix $S\in SO_n$, sodass $S^T A S$ eine diagonale Matrix mit Eigenwerten $\lambda_1\ldots\lambda_n$ von $A$ auf der Diagonale ist. Nach (a) ist dann $\sum\limits_{i,j=1}^n \left(A_{i,j}\right)^2 = \sum\limits_{i,j=1}^n \left((S^T A S)_{i,j}\right)^2 = \sum\limits_{k=1}^n \lambda_{k}^2\qed$
    \end{enumerate}
\end{task}
\begin{task}
    $f-g$ hat $n+1$ Nullstellen ($t_0,\dots,t_n$) in $\R$, ist aber maximal vom Grad $n$. Damit ist $\mathrm{deg}(f-g) < 1$ und damit $\mathrm{deg}(f-g) = 0$. Also sind $f$ und $g$ maximal um eine Konstante verschieden. Da aber $f$ und $g$ an mindestens einer Stelle gleich sind, ist $f=g$.
\end{task}
\end{document}