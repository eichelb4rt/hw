\documentclass[11pt]{article}
\usepackage{geometry}
\geometry{a4paper, top=20mm, left=10mm, right=10mm, bottom=20mm}
\usepackage{graphicx}
\usepackage{amsmath,amssymb,amsthm}
\usepackage{amssymb}
\usepackage[utf8]{inputenc}
\usepackage{fancyhdr}
\usepackage{lastpage}
\usepackage{enumerate}
\usepackage{enumitem}
\usepackage{multicol}
\usepackage{subcaption}
\usepackage{ifthen}
\usepackage{listings}
\usepackage{color}
\usepackage{scalerel}
%------------------------------------------ preamble
%----- fancyhdr
\fancyhead[R]{Übungsgruppe: 1 (Di 10-12)}
\fancyhead[C]{Name: Maurice Wenig}
\fancyhead[L]{Matrikelnummer: 178049}
\fancyfoot{}
\rfoot{Seite \thepage\ von \pageref{LastPage}}
\pagestyle{fancy}
%----- aufgaben
\newtheoremstyle{break}{}{5mm}{}{}{\bfseries}{}{0mm}
{\textbf{\thmname{#1}\thmnumber{ \hw.#2:} \thmnote{\textit{#3}}\newline}}
\theoremstyle{break}
\newtheorem{task}{Aufgabe}
%----- listings
\definecolor{mygreen}{rgb}{0,0.6,0}
\definecolor{mygray}{rgb}{0.5,0.5,0.5}
\definecolor{mymauve}{rgb}{0.58,0,0.82}
\lstset{ %
backgroundcolor=\color{white}, 
% choose the background color; you must add
basicstyle=\footnotesize, 
% the size of the fonts that are used for the code
breakatwhitespace=false, 
% sets if automatic breaks should only happen at whitespace
breaklines=true,
% sets automatic line breaking
captionpos=b,
% sets the caption-position to bottom
commentstyle=\color{mygreen},
% comment style
deletekeywords={...},
% if you want to delete keywords from the given language
extendedchars=true,
% lets you use non-ASCII characters; for 8-bits encodings only   
% adds a frame around the code
keepspaces=true, 
% keeps spaces in text, useful for keeping indentation of code
keywordstyle=\color{blue}, 
% keyword style
language=java, 
% the language of the code
otherkeywords={@Override},
% if you want to add more keywords to the set
numbers=left, 
% where to put the line-numbers; possible values(none, left, right)
numbersep=5pt, 
% how far the line-numbers are from the code
numberstyle=\tiny\color{mygray}, % the style that is used for the line-numbers
% if not set, frame-color can change on line-breaks to text-color
showspaces=false, 
% show spaces everywhere adding particular underscores;
showstringspaces=false, 
% underline spaces within strings only
showtabs=false, 
% show tabs within strings adding particular underscores
stepnumber=1, 
% the step between two line-numbers.
stringstyle=\color{mymauve}, 
% string literal style
tabsize=2,     
% sets default tabsize to 2 spaces
% show the filename of files included with \lstinputlisting;
emph={@Override},
emphstyle={\color{Goldenrod}}
}
%----- new commands
\newcommand{\Romannumeral}[1]{\MakeUppercase{\romannumeral #1}}
\newcommand{\set}[1]{\ensuremath{\{#1\}}}
\newcommand{\abs}[1]{\ensuremath{\left\vert #1 \right\vert}}
\newcommand{\norm}[1]{\ensuremath{\left\| #1 \right\|}}
\newcommand{\skal}[2]{\ensuremath{\left\langle #1 | #2 \right\rangle}}
\newcommand{\script}[1]{
    skripte/aufgabe#1.py
    \lstinputlisting{skripte/aufgabe#1.py}
}
%----- defs
\def\notiff{\mathrel{{\ooalign{\hidewidth$\not\phantom{"}$\hidewidth\cr$\iff$}}}}
\def\R{\ensuremath{\mathbb{R}}}
\def\1{\ensuremath{{\normalfont\hbox{1\kern-0.18em \vrule width .6pt}}}}
\def\ndy{
    \textcolor{red} {\hfill not done yet!}
    \reversemarginpar
    \marginpar{\raggedleft\textcolor{red}{\rule{2mm}{2mm}}}
}
%----- homework number
\newcommand{\hw}{4}
%------------------------------------------ main
\begin{document}
%----- title
\begin{center}
\Large{Numerische Mathematik}\\
\large{\hw. Übungsserie}
\end{center}
%----- tasks
\begin{task}
    \hfill\vspace{-5mm}
    \begin{enumerate}[label={(\alph*)}]
        \item $A\cdot F(k;\alpha_{k+1},\dots,\alpha_{m})=(\1 + F(k;\alpha_{k+1},\dots,\alpha_{m}) - \1) \cdot A = A + (F(k;\alpha_{k+1},\dots,\alpha_{m}) - \1) \cdot A$
        \begin{align*}
            F(k;\alpha_{k+1},\dots,\alpha_{m})_{i,j} - \1 &= \begin{cases}
                \alpha_{i} &: j=k\land i>k\\
                0 &: \text{sonst}
            \end{cases}\\
            \implies ((F(k;\alpha_{k+1},\dots,\alpha_{m}) - \1)\cdot A)_{i,j} &= \begin{cases}
                \alpha_{i}\cdot A_{k,j} &: i>k\\
                0 &: sonst
            \end{cases}
        \end{align*}
        \item $F\cdot F^{-1}$: jeweils das $\alpha_i$-Fache der $k$-ten Zeile wird zur $i$-ten Zeile von $F^{-1}$ addiert. Die $k$-te Zeile ist 1 in der $k$-ten Spalte, sonst 0. Damit ist das $\alpha_i$-Fache der $k$-ten Zeile $\alpha_i$. Das wird zur $i$-ten Zeile von $F^{-1}$ addiert, in der $-\alpha_i$ steht. $\implies F\cdot F^{-1}=\1$
        \item $F_{>k}:=F_{k+1}\cdot \ldots \cdot F_{m-1}$, $F_k\cdot F_{>k}$ ist jeweils das $\alpha_{i,k}$-Fache der $k$-ten Zeile von $F_{>k}$ zur $i$-ten Zeile von $F_{>k}$ addiert. $(F_{>k})_{k,k} = 1$, sonst ist $F_{>k}$ in der $k$-ten Zeile und Spalte 0. Also ist $F_k\cdot F_{>k} = F_{>k} (+) \alpha_{i,k}$ jeweils in der $i$-ten Zeile der $k$-ten Spalte. Da $F_{m-1} = F(m-1;\alpha_{m,m-1})$ ist $F_{>k}$ die normierte Dreiecksmatrix, deren Einträge unterhalb der Diagonale gegeben sind durch $\forall i>j>k:F_{i,j}=\alpha_{i,j}$. $F_{>0}=F$.
        \item Matrixmultiplikation auf $\R^{m\times m}$ ist assoziativ und hat das neutrale Element $\1_m$. $A,B\in L_m(\R)\implies (A\cdot B)_{i,j} = \sum\limits_{l=1}^{m} A_{i,l}\cdot B_{l,j}=\begin{cases}
            1 &: i=j\\
            0 &: i<j\\
            \text{irgendetwas} &: i>j
        \end{cases}\implies AB\in L_m(\R)$. Jedes $F\in L_m(\R)$ ist ein Produkt aus Frobenius-Matrizen $F_1\cdot\ldots\cdot F_{m-1}$ mit $F_k=F(k;F_{k+1,k},\dots,F_{m,k})$. Dann hat $F$ ein Inverses $F^{-1} = F^{-1}_{m-1}\cdot\ldots\cdot F^{-1}_{1}$: $F^{-1}\cdot F = F^{-1}_{m-1}\cdot\ldots\cdot F^{-1}_{1} \cdot F_1\cdot\ldots\cdot F_{m-1}=\1$ durch Assoziativität. Durch (b) und (c) ist $F^{-1}\in L_m(\R)$
    \end{enumerate}
\end{task}

\begin{task}
    \begin{align*}
        & \begin{pmatrix}
            6 & -2 & 2 & 4 \\
            12 & -8 & 6 & 10 \\
            3 & -13 & 9 & 3 \\
            -6 & 4 & 1 & -18
        \end{pmatrix} && 
        \begin{pmatrix}
            1 & 0 & 0 & 0 \\
            0 & 1 & 0 & 0 \\
            0 & 0 & 1 & 0 \\
            0 & 0 & 0 & 1 
        \end{pmatrix}\\
        & \begin{pmatrix}
            6 & -2 & 2 & 4 \\
            0 & -4 & 2 & 2 \\
            0 & -12 & 8 & 1 \\
            0 & 2 & 3 & -14
        \end{pmatrix} && 
        \begin{pmatrix}
            1 & 0 & 0 & 0 \\
            2 & 1 & 0 & 0 \\
            \frac{1}{2} & 0 & 1 & 0 \\
            -1 & 0 & 0 & 1 
        \end{pmatrix}\\
        & \begin{pmatrix}
            6 & -2 & 2 & 4 \\
            0 & -4 & 2 & 2 \\
            0 & 0 & 2 & -5 \\
            0 & 0 & 4 & -13
        \end{pmatrix} && 
        \begin{pmatrix}
            1 & 0 & 0 & 0 \\
            2 & 1 & 0 & 0 \\
            \frac{1}{2} & 3 & 1 & 0 \\
            -1 & -\frac{1}{2} & 0 & 1 
        \end{pmatrix}\\
        R &= \begin{pmatrix}
            6 & -2 & 2 & 4 \\
            0 & -4 & 2 & 2 \\
            0 & 0 & 2 & -5 \\
            0 & 0 & 0 & -3
        \end{pmatrix} & L &= 
        \begin{pmatrix}
            1 & 0 & 0 & 0 \\
            2 & 1 & 0 & 0 \\
            \frac{1}{2} & 3 & 1 & 0 \\
            -1 & -\frac{1}{2} & 2 & 1 
        \end{pmatrix}
    \end{align*}
\end{task}
\newpage
\begin{task}
    \hfill\vspace{-5mm}
    \begin{enumerate}[label={(\alph*)}]
        \item \hfill\vspace{-5mm}\\
        $\left(\begin{array}{cc|c}
            10^{-3} & 1 & 1\\
            1 & 1 & 2
        \end{array}\right)
        \rightarrow\left(\begin{array}{cc|c}
            10^{-3} & 1 & 1\\
            0 & -999 & -998
        \end{array}\right)
        \rightarrow LR(A;b)=\left\{\begin{array}{c}
            \frac{1000}{999}\vspace{1mm}\\
            \frac{998}{999}
        \end{array}\right\}$

        \item \hfill\vspace{-5mm}\\
        $\left(\begin{array}{cc|c}
            10^{-3} & 1 & 1\\
            1 & 1 & 2
        \end{array}\right)
        \rightarrow\left(\begin{array}{cc|c}
            10^{-3} & 1 & 1\\
            0 & -10^3 & -10^3
        \end{array}\right)
        \rightarrow LR(A;b)=\left\{\begin{array}{c}
            0\vspace{1mm}\\
            1
        \end{array}\right\}$

        \item \hfill\vspace{-5mm}\\
        $\left(\begin{array}{cc|c}
            1 & 1 & 2\\
            10^{-3} & 1 & 1
        \end{array}\right)
        \rightarrow\left(\begin{array}{cc|c}
            1 & 1 & 2\\
            0 & 1 & 1
        \end{array}\right)
        \rightarrow LR(A;b)=\left\{\begin{array}{c}
            1\vspace{1mm}\\
            1
        \end{array}\right\}$

        \item $A(\overline{x} + y) = A\overline{x} + Ay = b - A\overline{x} + A\overline{x} = b$\\
        $(b):\ \Delta = \left\{\begin{array}{c}
            0\vspace{1mm}\\
            1
        \end{array}\right\},\ \left(\begin{array}{cc|c}
            10^{-3} & 1 & 0\\
            1 & 1 & 1
        \end{array}\right)
        \rightarrow \left(\begin{array}{cc|c}
            10^{-3} & 1 & 0\\
            0 & -10^3 & 1
        \end{array}\right)
        \rightarrow LR(A;\Delta) = \left\{\begin{array}{c}
            1\vspace{1mm}\\
            -10^{-3}
        \end{array}\right\}\\$
        
        $(c):\ \Delta = \left\{\begin{array}{c}
            10^{-3}\vspace{1mm}\\
            0
        \end{array}\right\},\ \left(\begin{array}{cc|c}
            10^{-3} & 1 & 10^{-3}\\
            1 & 1 & 0
        \end{array}\right)
        \rightarrow \left(\begin{array}{cc|c}
            10^{-3} & 1 & 10^{-3}\\
            0 & -10^3 & 1
        \end{array}\right)
        \rightarrow LR(A;\Delta) = \left\{\begin{array}{c}
            2\vspace{1mm}\\
            -10^{-3}
        \end{array}\right\}$
    \end{enumerate}
\end{task}
\end{document}