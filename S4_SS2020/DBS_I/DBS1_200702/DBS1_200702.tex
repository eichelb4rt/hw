\documentclass[11pt]{article}
\usepackage{geometry}
\geometry{a4paper, top=20mm, left=10mm, right=10mm, bottom=20mm}
\usepackage{graphicx}
\usepackage{amsmath,amssymb,amsthm}
\usepackage{amssymb}
\usepackage[utf8]{inputenc}
\usepackage{fancyhdr}
\usepackage{lastpage}
\usepackage{enumerate}
\usepackage{enumitem}
\usepackage{multicol}
\usepackage{subcaption}
\usepackage{ifthen}
\usepackage{listings}
\usepackage{color}
\usepackage{scalerel}
\usepackage{tikz}
\usetikzlibrary{positioning, shapes, arrows}
%------------------------------------------ preamble
%----- fancyhdr
\fancyhead[L]{Name: Maurice Wenig}
\fancyhead[R]{Matrikelnummer: 178049}
\fancyfoot{}
\rfoot{Seite \thepage\ von \pageref{LastPage}}
\pagestyle{fancy}
%----- aufgaben
\newtheoremstyle{break}{}{5mm}{}{}{\bfseries}{}{0mm}
{\textbf{\thmname{#1}\thmnumber{ #2:} \thmnote{\textit{#3}}\newline}}
\theoremstyle{break}
\newtheorem{task}{Aufgabe}
%----- new commands
\newcommand{\Romannumeral}[1]{\MakeUppercase{\romannumeral #1}}
\newcommand{\notiff}{\mathrel{{\ooalign{\hidewidth$\not\phantom{"}$\hidewidth\cr$\iff$}}}}
\newcommand{\set}[1]{\ensuremath{\{#1\}}}
\newcommand{\abs}[1]{\ensuremath{\left\vert #1 \right\vert}}
\newcommand{\norm}[1]{\ensuremath{\left\| #1 \right\|}}
\newcommand{\skal}[2]{\ensuremath{\left\langle #1 | #2 \right\rangle}}
\newcommand{\R}{\ensuremath{\mathbb{R}}}
\newcommand{\ndy}{
    \textcolor{red} {\hfill not done yet!}
    \reversemarginpar
    \marginpar{\raggedleft\textcolor{red}{\rule{2mm}{2mm}}}
}
%------------------------------------------ main
\begin{document}
%----- title
\begin{center}
\Large{Datenbanksysteme I}\\
\large{8. Übungsserie}
\end{center}
%----- tasks
\begin{task}
	Es tut mir Leid.\\
	1:\vspace{-3cm}
	\begin{center}
	\includegraphics[scale=1]{"files/tree_1"}
	\end{center}
	2:\vspace{-3cm}
	\begin{center}
	\includegraphics[scale=1]{"files/tree_2"}
	\end{center}
\end{task}

\begin{task}
	Schlüssel + Daten sind 108 Byte groß $\implies$ in einem Knoten können maximal $\lfloor\frac{2^{16}\,Byte}{108\,Byte}\rfloor \approx	 \lfloor 151.7 \rfloor = 151$ Elemente sein. $\implies k = \lfloor 151/2 \rfloor = \underline{\underline{75}}$ 
\end{task}

\begin{task}
	\hfill\vspace{-5mm}
	\begin{enumerate}[label={(\alph*)}]
		\item 5
		\item Eine Festplatte darf auf einmal ausfallen. Beim Hinzufügen von Festplatten verändert sich nichts: Bei der gleichen Anzahl an Blöcken spielt die Anzahl der Festplatten keine Rolle. Die Paritätsbits müssen dann entsprechend alle Festplatten beinhalten.
		\item Ja, RAID6: Da benutzt man Striping mit mehreren verteilten Paritätsplatten
(kann mehrere Ausfälle überleben)
		\item \begin{align*}
		C &= 1000\\
		P_{D-F} &= 0101\\
		H &= 1110
		\end{align*}
	\end{enumerate}
\end{task}
\end{document}