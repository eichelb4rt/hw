\documentclass[11pt]{article}
\usepackage{geometry}
\geometry{a4paper, top=20mm, left=10mm, right=10mm, bottom=20mm}
\usepackage{graphicx}
\usepackage{amsmath,amssymb,amsthm}
\usepackage{amssymb}
\usepackage[utf8]{inputenc}
\usepackage{fancyhdr}
\usepackage{lastpage}
\usepackage{enumerate}
\usepackage{enumitem}
\usepackage{multicol}
\usepackage{subcaption}
\usepackage{ifthen}
\usepackage{listings}
\usepackage{color}
\usepackage{scalerel}
\usepackage{tikz}
\usetikzlibrary{positioning, shapes, arrows}
%------------------------------------------ preamble
%----- fancyhdr
\fancyhead[L]{Name: Maurice Wenig}
\fancyhead[R]{Matrikelnummer: 178049}
\fancyfoot{}
\rfoot{Seite \thepage\ von \pageref{LastPage}}
\pagestyle{fancy}
%----- aufgaben
\newtheoremstyle{break}{}{5mm}{}{}{\bfseries}{}{0mm}
{\textbf{\thmname{#1}\thmnumber{ #2:} \thmnote{\textit{#3}}\newline}}
\theoremstyle{break}
\newtheorem{task}{Aufgabe}
%----- new commands
\newcommand{\Romannumeral}[1]{\MakeUppercase{\romannumeral #1}}
\newcommand{\notiff}{\mathrel{{\ooalign{\hidewidth$\not\phantom{"}$\hidewidth\cr$\iff$}}}}
\newcommand{\set}[1]{\ensuremath{\{#1\}}}
\newcommand{\abs}[1]{\ensuremath{\left\vert #1 \right\vert}}
\newcommand{\norm}[1]{\ensuremath{\left\| #1 \right\|}}
\newcommand{\skal}[2]{\ensuremath{\left\langle #1 | #2 \right\rangle}}
\newcommand{\R}{\ensuremath{\mathbb{R}}}
\newcommand{\ndy}{
    \textcolor{red} {\hfill not done yet!}
    \reversemarginpar
    \marginpar{\raggedleft\textcolor{red}{\rule{2mm}{2mm}}}
}
%------------------------------------------ main
\begin{document}
%----- title
\begin{center}
\Large{Datenbanksysteme I}\\
\large{7. Übungsserie}
\end{center}
%----- tasks
\begin{task}
    \hfill\vspace{-5mm}
    \begin{enumerate}[label={(\alph*)}]
        \item \hfill\lstinputlisting[language=sql]{code/1a.txt}
        \item \hfill\lstinputlisting[language=sql]{code/1b.txt}
    \end{enumerate}
\end{task}

\begin{task}
    \hfill\vspace{-5mm}
    \begin{enumerate}[label={(\alph*)}]
        \item \begin{align*}
            (alles) &\rightarrow Verbund\\
            von &\rightarrow von\ GPS\\
            von\ GPS &\rightarrow von\\
            nach &\rightarrow nach\ GPS\\
            nach\ GPS &\rightarrow nach\\
            (Linie, von, nach) &\rightarrow Preis\\
            Linie &\rightarrow Modus\\
            Linie &\rightarrow \#Fahrzeuge
        \end{align*}
        \item $\set{Linie, [von / von\ GPS], [nach / nach\ GPS]}$ - mit den eckigen Klammern meine ich, dass eins von beiden genutzt werden kann.
    \end{enumerate}
\end{task}

\begin{task}
    \hfill\vspace{-5mm}
    \begin{enumerate}[label={(\alph*)}]
        \item \hfill\lstinputlisting[language=sql]{code/3a.txt}
        \item So viele, wie C unterschiedliche Elemente hat.
        \item Anomalie: die ganzen Tupel müssen gelöscht werden
        \item \begin{align*}
            \mathcal{R}_1 &= \set{\underline{person, kind\_name}, kind\_alter}\\
            \mathcal{R}_2 &= \set{\underline{person, fahrrad\_typ, fahrrad\_farbe}}
        \end{align*}
        \item \hfill\lstinputlisting[language=sql]{code/3e.txt}
    \end{enumerate}
\end{task}
\end{document}