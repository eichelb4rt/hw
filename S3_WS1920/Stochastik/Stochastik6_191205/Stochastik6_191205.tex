\documentclass[11pt]{article}
\usepackage{geometry}
\geometry{a4paper, top=20mm, left=10mm, right=10mm, bottom=20mm}
\usepackage{graphicx}
\usepackage{amsmath,amssymb,amsthm}
\usepackage{amssymb}
\usepackage[utf8]{inputenc}
\usepackage{fancyhdr}
\usepackage{lastpage}
\usepackage{enumerate}
\usepackage{enumitem}
\usepackage{multicol}
\usepackage{tikz}
\usepackage{subcaption}
\usepackage{rotating}
%------------------------------------------ preamble
%----- fancyhdr
\fancyhead[R]{Übungsgruppe: 4 (Do 12-14)}
\fancyhead[C]{Name: Maurice Wenig}
\fancyhead[L]{Matrikelnummer: 178049}
\fancyfoot{}
\rfoot{Seite \thepage\ von \pageref{LastPage}}
\pagestyle{fancy}
%----- aufgabena
\newtheoremstyle{break}{}{5mm}{}{}{\bfseries}{}{0mm}
{\textbf{\thmname{#1}\thmnumber{ #2:} \thmnote{\textit{#3}}\newline}}
\theoremstyle{break}
\newtheorem{task}{Aufgabe}
%----- newcommands
%------------------------------------------ main
\begin{document}
%----- title
\begin{center}
\Large{Einführung in die Wahrscheinlichkeitstheorie}\\
\large{Übungsserie 6}
\end{center}
%----- tasks
\setcounter{task}{1}
\begin{task}
    \hfill\vspace{-5mm}
    \begin{enumerate}[label={(\alph*)}]
        \item $\mathbb{P}((A\cap B)\cup(A\cap C)) = \mathbb{P}(A\cap B) + \mathbb{P}(A\cap C) - \mathbb{P}(A\cap B\cap C) = \frac{1}{9} + \frac{1}{9} - \frac{1}{27} = \underline{\underline{\frac{5}{27}}}$
        \item $\mathbb{P}(A\setminus (B\cup C)) = \mathbb{P}((A\cap B^C)\cup (A\cap C^C)) = \mathbb{P}(A\cap B^C) + \mathbb{P}(A\cap C^C) - \mathbb{P}(A\cap B^C\cap C^C) = \frac{2}{9} + \frac{2}{9} - \frac{4}{27} = \underline{\underline{\frac{8}{27}}}$
    \end{enumerate}
\end{task}
\begin{task}
    \hfill\vspace{-5mm}
    \begin{align*}
        \mathbb{P}(A) &= \frac{\binom{6}{4}}{2^6} = \frac{15}{2^6}\\
        \mathbb{P}(B) &= \frac{1}{2}\\
        \mathbb{P}(A\cap B) &= \frac{1}{2^2}\cdot \frac{1}{2^4} + \frac{1}{2^2}\cdot \frac{\binom{4}{2}}{2^4} = \frac{7}{2^6}\\
        &\neq \underline{\underline{\mathbb{P}(A)\mathbb{P}(B)}}
    \end{align*}
    $\implies$ Die beiden Ereignisse sind nicht unabhängig voneinander.
\end{task}
\setcounter{task}{4}
\begin{task}
    \hfill\vspace{-5mm}
    \begin{enumerate}[label={(\alph*)}]
        \item $\mathbb{P}(A\cap (B\cup C)) = \mathbb{P}((A\cap B)\cup (A\cap C)) = \mathbb{P}(A)\cdot \mathbb{P}(B) + \mathbb{P}(A)\cdot \mathbb{P}(C) - \underbrace{\mathbb{P}(A\cap B\cap C)}_{= 0\ (B\cap C = \emptyset)} \\= \mathbb{P}(A)(\mathbb{P}(B) + \mathbb{P}(C)) = \underline{\underline{\mathbb{P}(A)\mathbb{P}(B\cup C)}}$
        \item $\mathbb{P}(A\cap (B\cup C)) = \mathbb{P}(A)\mathbb{P}(B) + \mathbb{P}(A)\mathbb{P}(C) - \mathbb{P}(A)\mathbb{P}(B)\mathbb{P}(C) = \mathbb{P}(A)(\mathbb{P}(B) + \mathbb{P}(C) - \mathbb{P}(B)\mathbb{P}(C)) = \underline{\underline{\mathbb{P}(A)\mathbb{P}(B\cup C)}}$
    \end{enumerate}
\end{task}

\begin{task}
    \hfill\vspace{-5mm}
    \begin{enumerate}
        \item[(c)] $\mathbb{P}(A) = 1: \mathbb{P}(A\cap B) \stackrel{\mbox{\tiny ÜS 3}}{=} \mathbb{P}(B) = \underline{\underline{\mathbb{P}(A)\mathbb{P}(B)}}$\\
        $\mathbb{P}(A) = 0: \mathbb{P}(A\cap B) \stackrel{\mbox{\tiny ÜS 3}}{=} \mathbb{P}(A) = \underline{\underline{\mathbb{P}(A)\mathbb{P}(B)}}$
    \end{enumerate}
\end{task}
\end{document}