\documentclass[11pt]{article}
\usepackage{geometry}
\geometry{a4paper, top=20mm, left=10mm, right=10mm, bottom=20mm}
\usepackage{graphicx}
\usepackage{amsmath,amssymb,amsthm}
\usepackage{amssymb}
\usepackage[utf8]{inputenc}
\usepackage{fancyhdr}
\usepackage{lastpage}
\usepackage{enumerate}
\usepackage{enumitem}
\usepackage{multicol}
\usepackage{tikz}
\usepackage{subcaption}
\usepackage{rotating}
%------------------------------------------ preamble
%----- fancyhdr
\fancyhead[R]{Übungsgruppe: 4 (Do 12-14)}
\fancyhead[C]{Name: Maurice Wenig}
\fancyhead[L]{Matrikelnummer: 178049}
\fancyfoot{}
\rfoot{Seite \thepage\ von \pageref{LastPage}}
\pagestyle{fancy}
%----- aufgabena
\newtheoremstyle{break}{}{5mm}{}{}{\bfseries}{}{0mm}
{\textbf{\thmname{#1}\thmnumber{ #2:} \thmnote{\textit{#3}}\newline}}
\theoremstyle{break}
\newtheorem{task}{Aufgabe}
%----- newcommands
%------------------------------------------ main
\begin{document}
%----- title
\begin{center}
\Large{Einführung in die Wahrscheinlichkeitstheorie}\\
\large{Übungsserie 4}
\end{center}
%----- tasks
\begin{task}
    \hfill\vspace{-5mm}
    \begin{align*}
        \mathbb{P}(B\vert A) &= \frac{\mathbb{P}(B)\mathbb{P}(A\vert B)}{\mathbb{P}(B)\mathbb{P}(A\vert B) + \mathbb{P}(B^C)\mathbb{P}(A\vert B^C)}\\
        &= \frac{\mathbb{P}(B)\mathbb{P}(A\vert B)}{\mathbb{P}(B)\mathbb{P}(A\vert B) + \mathbb{P}(B^C)\mathbb{P}(A\vert B)}\\
        &= \frac{\mathbb{P}(B)\mathbb{P}(A\vert B)}{(\mathbb{P}(B) + \mathbb{P}(B^C))\mathbb{P}(A\vert B)}\\
        &= \underline{\underline{\mathbb{P}(B)}}\\
        &\implies \mathbb{P}(A\cap B) = \mathbb{P}(A)\mathbb{P}(B)
    \end{align*}
    
\end{task}
\setcounter{task}{2}
\begin{task}
    $B_0 = \{\text{Die ersten beiden Kugeln sind schwarz}\}\\
    B_1 = \{\text{Die erste Kugel ist schwarz, die andere weiß}\}\\
    B_2 = \{\text{Die erste Kugel ist weiß, die andere schwarz}\}\\
    B_3 = \{\text{Die ersten beiden Kugeln sind weiß}\}$
    \begin{enumerate}[label={(\alph*)}]
        \item \hfill\vspace{-5mm}\begin{align*}
            \mathbb{P}(A) &= \sum\limits_{i = 0}^{3} \mathbb{P}(B_i)\mathbb{P}(A\vert B_i)\\
            &= \frac{\binom{5}{1}}{\binom{9}{1}}\cdot \frac{\binom{4}{1}}{\binom{8}{1}}\cdot \frac{\binom{3}{1}}{\binom{7}{1}} &= \frac{60}{504}\\
            &\phantom{=} + \frac{\binom{5}{1}}{\binom{9}{1}}\cdot \frac{\binom{4}{1}}{\binom{8}{1}}\cdot \frac{\binom{4}{1}}{\binom{7}{1}} &\phantom{=} + \frac{80}{504}\\
            &\phantom{=} + \frac{\binom{4}{1}}{\binom{9}{1}}\cdot \frac{\binom{5}{5}}{\binom{8}{1}}\cdot \frac{\binom{4}{4}}{\binom{7}{1}} &\phantom{=} + \frac{80}{504}\\
            &\phantom{=} + \frac{\binom{4}{1}}{\binom{9}{1}}\cdot \frac{\binom{3}{3}}{\binom{8}{1}}\cdot \frac{\binom{2}{2}}{\binom{7}{1}} &\phantom{=} + \frac{24}{504}\\
            &&=\underline{\underline{\frac{244}{504}}}
        \end{align*}
        \item \hfill\vspace{-5mm}\begin{align*}
            \mathbb{P}(B\vert A) &= \frac{\mathbb{P}(A\vert B)\mathbb{P}(B)}{\mathbb{P}(A)}\\
            \mathbb{P}(B_0\vert A) &= \frac{\frac{\binom{5}{1}}{\binom{9}{1}}\cdot \frac{\binom{4}{1}}{\binom{8}{1}}\cdot \frac{\binom{3}{1}}{\binom{7}{1}}}{\mathbb{P}(A)} &= \underline{\underline{\frac{60}{244}}}\\
            \mathbb{P}((B_1\cup B_2)\vert A) &= \frac{\frac{\binom{5}{1}}{\binom{9}{1}}\cdot \frac{\binom{4}{1}}{\binom{8}{1}}\cdot \frac{\binom{4}{1}}{\binom{7}{1}} + \frac{\binom{4}{1}}{\binom{9}{1}}\cdot \frac{\binom{5}{5}}{\binom{8}{1}}\cdot \frac{\binom{4}{4}}{\binom{7}{1}}}{\mathbb{P}(A)} &= \underline{\underline{\frac{160}{244}}}\\
            \mathbb{P}(B_3\vert A) &= \frac{\frac{\binom{4}{1}}{\binom{9}{1}}\cdot \frac{\binom{3}{3}}{\binom{8}{1}}\cdot \frac{\binom{2}{2}}{\binom{7}{1}}}{\mathbb{P}(A)} &= \underline{\underline{\frac{24}{244}}}
        \end{align*}
    \end{enumerate}
\end{task}
\newpage
\begin{task}
    $A = \{\text{in Stau geraten}\}\\
    B_0 = \{\text{über A4 gefahren}\}\\
    B_1 = \{\text{über B7 gefahren}\}\\
    B_2 = \{\text{über L2161 gefahren}\}$
    \begin{enumerate}[label={(\alph*)}]
        \item \hfill\vspace{-5mm}\begin{align*}
            \mathbb{P}(A) &= \sum\limits_{i = 0}^{2} \mathbb{P}(B_i)\mathbb{P}(A\vert B_i)\\
            &= \frac{1}{4}\cdot\frac{1}{2} + \frac{1}{10}\cdot\frac{1}{3} + \frac{1}{20}\cdot\frac{1}{6} = \underline{\frac{1}{6}}\\
            \implies \mathbb{P}(A^C) &= \underline{\underline{\frac{5}{6}}}
        \end{align*}
        \item \hfill\vspace{-5mm}\begin{align*}
            \mathbb{P}(B\vert A) &= \frac{\mathbb{P}(A\vert B)\mathbb{P}(B)}{\mathbb{P}(A)}\\
            \mathbb{P}(B_0\vert A) &= \frac{\frac{1}{4}\cdot\frac{1}{2}}{\frac{1}{6}} &= \underline{\underline{\frac{3}{4}}}\\
            \mathbb{P}((B_1)\vert A) &= \frac{\frac{1}{10}\cdot\frac{1}{3}}{\frac{1}{6}} &= \underline{\underline{\frac{1}{5}}}\\
            \mathbb{P}(B_2\vert A) &= \frac{\frac{1}{20}\cdot\frac{1}{6}}{\frac{1}{6}} &= \underline{\underline{\frac{1}{20}}}
        \end{align*}
    \end{enumerate}
\end{task}
\end{document}