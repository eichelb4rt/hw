\documentclass[11pt]{article}
\usepackage{geometry}
\geometry{a4paper, top=20mm, left=10mm, right=10mm, bottom=20mm}
\usepackage{graphicx}
\usepackage{amsmath,amssymb,amsthm}
\usepackage{amssymb}
\usepackage[utf8]{inputenc}
\usepackage{fancyhdr}
\usepackage{lastpage}
\usepackage{enumerate}
\usepackage{enumitem}
\usepackage{algorithmicx}
\usepackage[noend]{algpseudocode}
\usepackage[plain]{algorithm}
\usepackage{multicol}
\usepackage{tikz}
\usepackage{subcaption}
\usepackage{rotating}
%------------------------------------------ preamble
%----- fancyhdr
\fancyhead[R]{Übungsgruppe: 4 (Do 12-14)}
\fancyhead[C]{Name: Maurice Wenig}
\fancyhead[L]{Matrikelnummer: 178049}
\fancyfoot{}
\rfoot{Seite \thepage\ von \pageref{LastPage}}
\pagestyle{fancy}
%----- aufgaben
\newtheoremstyle{break}{}{5mm}{}{}{\bfseries}{}{0mm}
{\textbf{\thmname{#1}\thmnumber{ #2:} \thmnote{\textit{#3}}\newline}}
\theoremstyle{break}
\newtheorem{task}{Aufgabe}
%----- newcommands
%------------------------------------------ main
\begin{document}
%----- title
\begin{center}
\Large{Einführung in die Wahrscheinlichkeitstheorie}\\
\large{Übungsserie 1}
\end{center}
%----- tasks
\setcounter{task}{1}
\begin{task}
\hfill\vspace{-5mm}
\begin{enumerate}[label={(\alph*)}]
\item $A\setminus (B\cup C) = A\cap (B\cup C)^C = A\cap B^C\cap C^C = (A\cap B^C)\cap (A\cap C^C) = \underline{\underline{(A\setminus B)\cap (A\setminus C)}}$
\item $A\cup\bigcap\limits_{i=1}^k B_i = \left( A^C\cap\left(\bigcap\limits_{i=1}^k B_i \right)^C \right)^C = \left( A^C\cap\left(\bigcup\limits_{i=1}^k B_i^C \right) \right)^C = \left(\bigcup\limits_{i=1}^k A^C\cap B_i^C \right)^C = \bigcap\limits_{i=1}^k\left( A^C\cap B_i^C \right)^C\\ = \bigcap\limits_{i=1}^k\left( A^C\cap B_i^C \right)^C = \underline{\underline{\bigcap\limits_{i=1}^k\left( A\cup B_i \right)}}$
\end{enumerate}
\end{task}
\setcounter{task}{5}
\begin{task}
\hfill\vspace{-5mm}
\begin{flalign*}
B_1 &= \bigcap\limits_{j=1}^{n}A_j,\ \left\vert B_1\right\vert = 1 &\\
B_2 &= \bigcup\limits_{j=1}^{n}A_j,\ \left\vert B_2\right\vert = 2^n-1 &\\
B_3 &= \bigcup\limits_{j=1}^{n}\left(A_j\cap\bigcap\limits_{i=1}^{n}A_i^C\mid i\neq j\right),\ \left\vert B_3\right\vert = n &\\
B_4 &= \bigcap\limits_{j=1}^{n}A_j \cup \bigcap\limits_{j=1}^{n}A_j^C,\ \left\vert B_4\right\vert = 2 &\\
\end{flalign*}
\end{task}
\end{document}