\documentclass[11pt]{article}
\usepackage{geometry}
\geometry{a4paper, top=20mm, left=10mm, right=10mm, bottom=20mm}
\usepackage{graphicx}
\usepackage{amsmath,amssymb,amsthm}
\usepackage{amssymb}
\usepackage[utf8]{inputenc}
\usepackage{fancyhdr}
\usepackage{lastpage}
\usepackage{enumerate}
\usepackage{enumitem}
\usepackage{multicol}
\usepackage{tikz}
\usepackage{subcaption}
\usepackage{rotating}
%------------------------------------------ preamble
%----- fancyhdr
\fancyhead[R]{Übungsgruppe: 4 (Do 12-14)}
\fancyhead[C]{Name: Maurice Wenig}
\fancyhead[L]{Matrikelnummer: 178049}
\fancyfoot{}
\rfoot{Seite \thepage\ von \pageref{LastPage}}
\pagestyle{fancy}
%----- aufgabena
\newtheoremstyle{break}{}{5mm}{}{}{\bfseries}{}{0mm}
{\textbf{\thmname{#1}\thmnumber{ #2:} \thmnote{\textit{#3}}\newline}}
\theoremstyle{break}
\newtheorem{task}{Aufgabe}
%----- newcommands
%------------------------------------------ main
\begin{document}
%----- title
\begin{center}
\Large{Einführung in die Wahrscheinlichkeitstheorie}\\
\large{Übungsserie 4}
\end{center}
%----- tasks
\setcounter{task}{2}
\begin{task}
\hfill\vspace{-5mm}
\begin{align*}
A &= \{\text{alle 3 Farben gezogen}\}\\
A^C &= B_{weiss} \cup B_{rot} \cup B_{schwarz}\\
B_i &= \{\text{Farbe }i\text{ nicht gezogen}\}\\
\mathbb{P}(A^C) &=
 \mathbb{P}(B_{weiss}) + \mathbb{P}(B_{rot}) + \mathbb{P}(B_{schwarz})\\
&\phantom{=}  - (\mathbb{P}(B_{weiss}\cap B_{rot}) + \mathbb{P}(B_{weiss}\cap B_{schwarz}) + \mathbb{P}(B_{schwarz}\cap B_{rot}))\\
&\phantom{=} + \mathbb{P}(B_{weiss} \cap B_{rot} \cap B_{schwarz})\\
\mathbb{P}(A^C) &= \frac{\binom{13}{5}}{\binom{20}{5}} + \frac{\binom{12}{5}}{\binom{20}{5}} + \frac{\binom{15}{5}}{\binom{20}{5}}\\
&\phantom{=}\ - \left(\frac{\binom{5}{5}}{\binom{20}{5}} + \frac{\binom{8}{5}}{\binom{20}{5}} + \frac{\binom{7}{5}}{\binom{20}{5}}\right)\\
&\phantom{=}\ + 0\\
\mathbb{P}(A) &= \underline{\underline{1-\mathbb{P}(A^C)}}
\end{align*}
\end{task}
\setcounter{task}{4}
\begin{task}
\hfill\vspace{-5mm}
\begin{enumerate}[label={(\alph*)}]
\item $\Omega = \{(w,w,u_i),(w,s,u_i),(s,w,u_i),(s,s,u_i)\mid i\in [1,3]\}\setminus \{(s,s,u_1)\}$
\item \hfill\vspace{-5mm}\begin{align*}
U_i &= \{\text{Urne } i \text{ wurde ausgewählt}\}\\
A_{xy} &= \{\text{Farbe } x\text{, dann }y \text{ wurde gezogen}\}\\\\
\mathbb{P}(A_{ww}) &= \mathbb{P}(U_1)\mathbb{P}(A_{ww}\vert U_1) + \mathbb{P}(U_2)\mathbb{P}(A_{ww}\vert U_2) + \mathbb{P}(U_3)\mathbb{P}(A_{ww}\vert U_3)\\
&= \frac{1}{3}(\mathbb{P}(A_{ww}\vert U_1) + \mathbb{P}(A_{ww}\vert U_2) + \mathbb{P}(A_{ww}\vert U_3))\\
&= \underline{\frac{1}{3}\left(\frac{\binom{2}{5}}{\binom{2}{6}} + \frac{\binom{2}{4}}{\binom{2}{6}} + \frac{\binom{2}{5}}{\binom{2}{6}}\right)}\\\\
\mathbb{P}(A_{ss}) &= \frac{1}{3}(\mathbb{P}(A_{ss}\vert U_1) + \mathbb{P}(A_{ss}\vert U_2) + \mathbb{P}(A_{ss}\vert U_3))\\
&= \underline{\frac{1}{3}\left(0 + \frac{\binom{2}{2}}{\binom{2}{6}} + \frac{\binom{2}{3}}{\binom{2}{6}}\right)}\\\\
\mathbb{P}(A_{ws}\cup A_{sw})&= \mathbb{P}(A_{ws}) + \mathbb{P}(A_{sw})\\
&= \frac{1}{3}\left(\frac{\binom{1}{5}}{\binom{1}{6}}\cdot \frac{\binom{1}{1}}{\binom{1}{5}} + \frac{\binom{1}{4}}{\binom{1}{6}}\cdot \frac{\binom{1}{2}}{\binom{1}{5}} + \frac{\binom{1}{3}}{\binom{1}{6}}\cdot \frac{\binom{1}{3}}{\binom{1}{5}}\right)\\
&\phantom{=}\ + \frac{1}{3}\left(\frac{\binom{1}{1}}{\binom{1}{6}}\cdot \frac{\binom{1}{5}}{\binom{1}{5}} + \frac{\binom{1}{2}}{\binom{1}{6}}\cdot \frac{\binom{1}{4}}{\binom{1}{5}} + \frac{\binom{1}{3}}{\binom{1}{6}}\cdot \frac{\binom{1}{3}}{\binom{1}{5}}\right)\\
&= \underline{\frac{2}{3}\left(\frac{\binom{1}{5}}{\binom{1}{6}}\cdot \frac{\binom{1}{1}}{\binom{1}{5}} + \frac{\binom{1}{4}}{\binom{1}{6}}\cdot \frac{\binom{1}{2}}{\binom{1}{5}} + \frac{\binom{1}{3}}{\binom{1}{6}}\cdot \frac{\binom{1}{3}}{\binom{1}{5}}\right)}
\end{align*}
\end{enumerate}
\end{task}
\end{document}