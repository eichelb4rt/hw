\documentclass[11pt]{article}
\usepackage{geometry}
\geometry{a4paper, top=20mm, left=10mm, right=10mm, bottom=20mm}
\usepackage{graphicx}
\usepackage{amsmath,amssymb,amsthm}
\usepackage{amssymb}
\usepackage[utf8]{inputenc}
\usepackage{fancyhdr}
\usepackage{lastpage}
\usepackage{enumerate}
\usepackage{enumitem}
\usepackage{multicol}
\usepackage{tikz}
\usepackage{subcaption}
\usepackage{rotating}
%------------------------------------------ preamble
%----- fancyhdr
\fancyhead[R]{Übungsgruppe: 4 (Do 12-14)}
\fancyhead[C]{Name: Maurice Wenig}
\fancyhead[L]{Matrikelnummer: 178049}
\fancyfoot{}
\rfoot{Seite \thepage\ von \pageref{LastPage}}
\pagestyle{fancy}
%----- aufgaben
\newtheoremstyle{break}{}{5mm}{}{}{\bfseries}{}{0mm}
{\textbf{\thmname{#1}\thmnumber{ #2:} \thmnote{\textit{#3}}\newline}}
\theoremstyle{break}
\newtheorem{task}{Aufgabe}
%----- newcommands
%------------------------------------------ main
\begin{document}
%----- title
\begin{center}
\Large{Einführung in die Wahrscheinlichkeitstheorie}\\
\large{Übungsserie 3}
\end{center}
%----- tasks
\begin{task}
    \hfill\vspace{-5mm}
    \begin{align*}
        \mathbb{P}(A^C)             &= 1 - \mathbb{P}(A) = \underline{\underline{\frac{3}{4}}}\\
        \mathbb{P}(A^C\cup B)       &= \mathbb{P}((A\cap B^C)^C) = \mathbb{P}((A\setminus B)^C) = 1 - \left(\frac{1}{3} - \frac{1}{6}\right) = \underline{\underline{\frac{5}{6}}}\\
        \mathbb{P}(A\cup B^C)       &= \mathbb{P}((B\setminus A)^C) = 1 - \left(\frac{1}{4} - \frac{1}{6}\right) = \underline{\underline{\frac{11}{12}}}\\
        \mathbb{P}(A\cap B^C)       &= \mathbb{P}(A\setminus B) = \frac{1}{3} - \frac{1}{6} = \underline{\underline{\frac{1}{6}}}\\
        \mathbb{P}(A\triangle B)    &= \mathbb{P}((A\setminus B)\cup (B\setminus A)) = \left(\frac{1}{3} - \frac{1}{6}\right) + \left(\frac{1}{4} - \frac{1}{6}\right) = \underline{\underline{\frac{1}{4}}}\\
        \mathbb{P}(A^C\cup B^C)     &= \mathbb{P}((A\cap B)^C) = 1-\frac{1}{6} = \underline{\underline{\frac{5}{6}}}\\
    \end{align*}
\end{task}

\setcounter{task}{2}
\begin{task}
\hfill\vspace{-5mm}
\begin{enumerate}[label={(\alph*)}]
\item \hfill\vspace{-8mm}
\begin{align*}
0\leq \mathbb{P}(A\cap B) \leq \mathbb{P}(B) &= 0 \implies \mathbb{P}(A\cap B) = 0&\\
\mathbb{P}(A\cup B) &= \mathbb{P}(A) + \underbrace{\mathbb{P}(B)}_{=0} - \underbrace{\mathbb{P}(A\cap B)}_{=0} = \underline{\underline{\mathbb{P}(A)}}&\\
\mathbb{P}(A\cup B) &= \mathbb{P}(A\setminus B) + \mathbb{P}(B\setminus A) + \mathbb{P}(A\cap B)&\\
&= \underline{\mathbb{P}(A\setminus B)} + \underbrace{\mathbb{P}(B\cap A^C)}_{=0} + \underbrace{\mathbb{P}(A\cap B)}_{=0} = \underline{\underline{\mathbb{P}(A)}}&\tag{1}
\end{align*}
\item \hfill\vspace{-8mm}
\begin{align*}
    \mathbb{P}(B^C)&=1-\mathbb{P}(B)=0&\\
    \mathbb{P}(A\cap B)&=P(A\setminus B^C) \stackrel{\mbox{\tiny (1)}}{=} \underline{\underline{\mathbb{P}(A)}}&\tag{2}\\
    \mathbb{P}(B\setminus A)&=P(B\cap A^C) \stackrel{\mbox{\tiny (2)}}{=} \underline{\underline{\mathbb{P}(A^C)}}
\end{align*}
\end{enumerate}
\end{task}

\setcounter{task}{4}
\begin{task}
     \hfill\vspace{-8mm}
    \begin{align*}
        \left\vert A \right\vert &= (n-1)! \\
        \left\vert \Omega \right\vert &= n! \\
        \mathbb{P}(A) &= \frac{\left\vert A \right\vert}{\left\vert \Omega \right\vert} = \underline{\underline{n}} 
    \end{align*}
\end{task}
\end{document}