\documentclass[11pt]{article}
\usepackage{geometry}
\geometry{a4paper, top=20mm, left=10mm, right=10mm, bottom=20mm}
\usepackage{graphicx}
\usepackage{amsmath,amssymb,amsthm}
\usepackage{amssymb}
\usepackage[utf8]{inputenc}
\usepackage{fancyhdr}
\usepackage{lastpage}
\usepackage{enumerate}
\usepackage{enumitem}
\usepackage{multicol}
\usepackage{tikz}
\usepackage{subcaption}
\usepackage{rotating}
%------------------------------------------ preamble
%----- fancyhdr
\fancyhead[R]{Übungsgruppe: 4 (Do 12-14)}
\fancyhead[C]{Name: Maurice Wenig}
\fancyhead[L]{Matrikelnummer: 178049}
\fancyfoot{}
\rfoot{Seite \thepage\ von \pageref{LastPage}}
\pagestyle{fancy}
%----- aufgabena
\newtheoremstyle{break}{}{5mm}{}{}{\bfseries}{}{0mm}
{\textbf{\thmname{#1}\thmnumber{ #2:} \thmnote{\textit{#3}}\newline}}
\theoremstyle{break}
\newtheorem{task}{Aufgabe}
%----- newcommands
%------------------------------------------ main
\begin{document}
%----- title
\begin{center}
\Large{Einführung in die Wahrscheinlichkeitstheorie}\\
\large{Übungsserie 7}
\end{center}
%----- tasks
\setcounter{task}{1}
\begin{task}
    \hfill\vspace{-5mm}
    \begin{enumerate}[label={(\alph*)}]
        \item $F(w) = \begin{cases}
            \frac{6}{36} & X(w) = 0\\
            \frac{10}{36} & X(w) = 1\\
            \frac{8}{36} & X(w) = 2\\
            \frac{6}{36} & X(w) = 3\\
            \frac{4}{36} & X(w) = 4\\
            \frac{2}{36} & X(w) = 5\\
            0 & \text{sonst}
        \end{cases}$
    \end{enumerate}
\end{task}
\end{document}