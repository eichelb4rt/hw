\documentclass[11pt]{article}
\usepackage{geometry}
\geometry{a4paper, top=20mm, left=10mm, right=10mm, bottom=20mm}
\usepackage{graphicx}
\usepackage{amsmath,amssymb,amsthm}
\usepackage{amssymb}
\usepackage[utf8]{inputenc}
\usepackage{fancyhdr}
\usepackage{lastpage}
\usepackage{enumerate}
\usepackage{enumitem}
\usepackage{multicol}
\usepackage{tikz}
\usepackage{subcaption}
\usepackage{rotating}
%------------------------------------------ preamble
%----- fancyhdr
\fancyhead[R]{Übungsgruppe: 4 (Do 12-14)}
\fancyhead[C]{Name: Maurice Wenig}
\fancyhead[L]{Matrikelnummer: 178049}
\fancyfoot{}
\rfoot{Seite \thepage\ von \pageref{LastPage}}
\pagestyle{fancy}
%----- aufgabena
\newtheoremstyle{break}{}{5mm}{}{}{\bfseries}{}{0mm}
{\textbf{\thmname{#1}\thmnumber{ #2:} \thmnote{\textit{#3}}\newline}}
\theoremstyle{break}
\newtheorem{task}{Aufgabe}
%----- newcommands
\newcommand{\set}[1]{\ensuremath{\{#1\}}}
%------------------------------------------ main
\begin{document}
%----- title
\begin{center}
\Large{Einführung in die Wahrscheinlichkeitstheorie}\\
\large{Übungsserie 7}
\end{center}
%----- tasks
\setcounter{task}{1}
\begin{task}
    $\mathbb{P}(X=k) = \begin{cases}
        \frac{6}{36} & k = 0\\
        \frac{10}{36} & k = 1\\
        \frac{8}{36} & k = 2\\
        \frac{6}{36} & k = 3\\
        \frac{4}{36} & k = 4\\
        \frac{2}{36} & k = 5\\
        0 & \text{sonst}
    \end{cases}
    \implies F(s) = \begin{cases}
        0 & X < 0\\
        \frac{6}{36} & 0 \leq s < 1\\
        \frac{16}{36} & 1 \leq s < 2\\
        \frac{24}{36} & 2 \leq s < 3\\
        \frac{30}{36} & 3 \leq s < 4\\
        \frac{34}{36} & 4 \leq s < 5\\
        1 & 5 \leq s
    \end{cases}$
\end{task}
\setcounter{task}{3}
\begin{task}
    \hfill\vspace{-5mm}
    \begin{enumerate}[label={(\alph*)}]
        \item \hfill\vspace{-5mm}\\$\mathbb{P}(X=t) = \begin{cases}
            \frac{1}{3} & t = 0\\
            \frac{1}{9} & t = 1\\
            \frac{1}{3} & t = 2\\
            \frac{2}{9} & t = 3\\
            0 & \text{sonst}
        \end{cases}$
        \item \hfill\vspace{-5mm}\begin{flalign*}
            \mathbb{P}\{0.5 < X \leq 2\} &= F(2) - F(0.5) = \frac{4}{9}&\\
            \mathbb{P}\{X < 2\} &= \mathbb{P}(\{ X \leq 2 \}\setminus \{ X = 2\}) = F(2) - \mathbb{P}(2) = \frac{4}{9}&\\
            \mathbb{P}\{X > 1.5\} &= 1- F(1.5) = \frac{5}{9}&\\
        \end{flalign*}
    \end{enumerate}
\end{task}

\begin{task}
    \hfill\vspace{-5mm}
    \begin{enumerate}[label={(\alph*)}]
        \item $\mathbb{P}\set{X=0} = 1 - \mathbb{P}\set{X>0} = 1 - \sum\limits_{k=1}^{\infty}\frac{1}{3^k} = 1 - \left(\frac{1}{1-\frac{1}{3}}-1\right) = \underline{\underline{\frac{1}{2}}}$
        \item $\mathbb{P}\set{X\geq m} = \sum\limits_{k=m}^{\infty}\frac{1}{3^k} = \sum\limits_{k=1}^{\infty}\frac{1}{3^k} - \sum\limits_{k=1}^{m-1}\frac{1}{3^k} = \underline{\underline{\frac{1}{2} - \sum\limits_{k=1}^{m-1}\frac{1}{3^k}}}$
        \item $\mathbb{P}\set{X\in A} = \sum\limits_{k=1}^{\infty}\frac{1}{3^{2k}} = \sum\limits_{k=1}^{\infty}\frac{1}{9^k} = \frac{1}{1 - \frac{1}{9}} - 1 = \underline{\underline{\frac{1}{8}}}$
    \end{enumerate}
\end{task}
\end{document}