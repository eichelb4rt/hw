\documentclass[11pt]{article}
\usepackage{geometry}
\geometry{a4paper, top=20mm, left=10mm, right=10mm, bottom=20mm}
\usepackage{graphicx}
\usepackage{amsmath,amssymb,amsthm}
\usepackage{amssymb}
\usepackage[utf8]{inputenc}
\usepackage{fancyhdr}
\usepackage{lastpage}
\usepackage{enumerate}
\usepackage{enumitem}
\usepackage{tikz}
\usetikzlibrary{shapes.geometric}
\usepackage{multicol}
\usepackage{subcaption}
\usepackage{ifthen}
%------------------------------------------ preamble
%----- fancyhdr
\fancyhead[R]{Übungsgruppe: 2 (Do 12-14)}
\fancyhead[C]{Name: Maurice Wenig}
\fancyhead[L]{Matrikelnummer: 178049}
\fancyfoot{}
\rfoot{Seite \thepage\ von \pageref{LastPage}}
\pagestyle{fancy}
%----- aufgaben
\newtheoremstyle{break}{}{5mm}{}{}{\bfseries}{}{0mm}
{\textbf{\thmname{#1}\thmnumber{ #2:} \thmnote{\textit{#3}}\newline}}
\theoremstyle{break}
\newtheorem{task}{Aufgabe}
%----- new commands
\newcommand{\Romannumeral}[1]{\MakeUppercase{\romannumeral #1}}
%----- tikz automata
\usetikzlibrary{arrows, automata, positioning}
%------------------------------------------ main
\begin{document}
%----- title
\begin{center}
\Large{Automaten und Berechenbarkeit}\\
\large{5. Übungsserie}
\end{center}
%----- tasks

\begin{task}
    \hfill\vspace{-5mm}
    \begin{enumerate} [label={(\alph*)}]
        \item Nein, Pumping Lemma: $n\in \mathbb{N},\ z = a^n \# a^{f(n)}\\ uv = a^n,\ v = a^m,\ uv^0 w = a^{n-m}\# a^{f(n)},\ f(n)\neq f(n-m)(\text{Injektivität})\implies uv^0 w \notin L_f$
        \item Ja, $L_1 = \{ a^{2(m+n)}\mid m,n\in \mathbb{N} \} = L(0,2)$
        \item Nein, Pumping Lemma: $n\in \mathbb{N},\ z = a^n\# a^m\# a^{m+n}\\ uv = a^n,\ v = a^x,\ uv^0 w = a^{n-x}\# a^m\# a^{n+m}\notin L$
        \item Nein, Pumping Lemma: $n\in \mathbb{N},\ z = a^n\# a^m\# a^{m+n}\\ uv = a^n,\ v = a^x,\ uv^0 w = a^{n-x}\# a^m\# a^{n+m}\notin L,\text{ da } m+n > m+n-x$
        \item Ja, $A = (Q, \Sigma, \delta, q_0, F),\ L = L(A) \vspace{2mm}\\
        Q = \{ d_i\mid 0\leq i\leq d-1 \}\cup \{ c_{i}\mid 0\leq i\leq c-1 \}\\
        \Sigma = \{a\}\\
        \delta(q, a) := \begin{cases}
            c_{i+1} & \text{falls } q = c_i, i\neq c-1\\
            d_{i+1} & \text{falls } q = d_i, i\neq d-1\\
            d_{0} & \text{falls } q = c_{c-1}\text{ oder } q = d_{d-i}
        \end{cases}\\
        q_0 = c_0 \\
        F = \{ d_0 \}\vspace{2mm}$
    \end{enumerate}
\end{task}

\begin{task}
    $F = \underbrace{\{\bigcup_{m\in \mathbb{N}} 0^{m+n} 1^{m}\mid n\in\mathbb{N}\}}_{\text{Wörter, die noch } n \text{ Einsen brauchen}}\cup \underbrace{\{\Sigma^*\setminus \{ 0^{m+n} 1^{m}\mid m,n\in\mathbb{N}\}\}}_{\text{Wörter, die nicht mehr in } L \text{ sein können}}$\\
    Da der Schnitt zweier Äquivalenzklassen leer ist und die Vereinigung aller Äquivalenzklassen $\Sigma^*$ darstellt,\\ist $F$ eine Zerlegung von $\Sigma^*$.
\end{task}

\begin{task}
    $F = \{\{w\}\mid w\in \Sigma^*\}$: Jedes Wort hat seine eigene Äquivalenzklasse, da nur es selbst in der Sprache bleibt, wenn das dazugehörige Spiegelwort angehangen wird. Da der Schnitt zweier Äquivalenzklassen leer ist und die Vereinigung aller Äquivalenzklassen $\Sigma^*$ darstellt, ist $F$ eine Zerlegung von $\Sigma^*$.
\end{task}

\begin{task}
    \hfill\vspace{-5mm}
    \begin{enumerate} [label={(\alph*)}]
        \item folgt aus (b)
        \item $n\in \mathbb{N},\ z = a^p\text{ wobei } p \text{ die kleinste Primzahl mit } p\geq n \text{ sei}\\ uv = a^n,\ v = a^m,\ \{uv^i w \mid i\in \mathbb{N}\} = L(p-m, m)\nsubseteq L_{prim}(1.a)\implies \exists_{i\in\mathbb{N}}: uv^i w \notin L_{prim}$
        \item Nein, Pumping Lemma analog zu (b)
        \item Ja, $L_{prim}^* = \{ a^n\mid n\in\mathbb{N},\ n\geq 2 \}\cup\{\lambda\} = \Sigma^*\setminus\{a\}$, da jedes $n\in \mathbb{N},\ n\geq 2$ Primfaktoren hat und somit auch als Summe von Primzahlen darstellbar ist.
    \end{enumerate}
\end{task}
\end{document}
