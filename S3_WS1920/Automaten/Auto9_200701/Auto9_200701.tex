\documentclass[11pt]{article}
\usepackage{geometry}
\geometry{a4paper, top=20mm, left=10mm, right=10mm, bottom=20mm}
\usepackage{graphicx}
\usepackage{amsmath,amssymb,amsthm}
\usepackage{amssymb}
\usepackage[utf8]{inputenc}
\usepackage{fancyhdr}
\usepackage{lastpage}
\usepackage{enumerate}
\usepackage{enumitem}
\usepackage{tikz}
\usetikzlibrary{shapes.geometric}
\usepackage{multicol}
\usepackage{subcaption}
\usepackage{ifthen}
%------------------------------------------ preamble
%----- fancyhdr
\fancyhead[R]{Übungsgruppe: 2 (Do 12-14)}
\fancyhead[C]{Name: Maurice Wenig}
\fancyhead[L]{Matrikelnummer: 178049}
\fancyfoot{}
\rfoot{Seite \thepage\ von \pageref{LastPage}}
\pagestyle{fancy}
%----- aufgaben
\newtheoremstyle{break}{}{5mm}{}{}{\bfseries}{}{0mm}
{\textbf{\thmname{#1}\thmnumber{ #2:} \thmnote{\textit{#3}}\newline}}
\theoremstyle{break}
\newtheorem{task}{Aufgabe}
%----- new commands
\newcommand{\Romannumeral}[1]{\MakeUppercase{\romannumeral #1}}
\newcommand{\notiff}{\mathrel{{\ooalign{\hidewidth$\not\phantom{"}$\hidewidth\cr$\iff$}}}}
\newcommand{\set}[1]{\ensuremath{\{#1\}}}
\newcommand{\abs}[1]{\ensuremath{\left\vert #1 \right\vert}}
%----- tikz automata
\usetikzlibrary{arrows, automata, positioning}
%------------------------------------------ main
\begin{document}
%----- title
\begin{center}
\Large{Automaten und Berechenbarkeit}\\
\large{9. Übungsserie}
\end{center}
%----- tasks
\begin{task}
    \hfill\vspace{-5mm}
    \begin{enumerate}[label={(\alph*)}]
        \item $G_a = (N_a, T, S_a, P_a)$\vspace{-2mm}
        \begin{flalign*}
            N_a &= \set{Q_0, Q_1, Q_2, Q_3, Q_4} &\\
            T &= \set{a,b} &\\
            S_a &= Q_0 &\\
            P_a &= \set{Q_0\to \lambda, Q_0\to aQ_1, Q_0\to bQ_0, Q_1\to aQ_2, Q_1\to bQ_1, Q_2\to aQ_3, Q_2\to bQ_2,&\\&\phantom{=\{ }\ Q_3\to aQ_4, Q_3\to bQ_3, Q_4\to aQ_0, Q_4\to bQ_4} &
        \end{flalign*}
        $G_{a2} = (N_{a2}, T, S_{a2}, P_{a2})$\vspace{-2mm}
        \begin{flalign*}
            N_{a2} &= \set{Q_0, Q_1, Q_2, Q_3, Q_4} &\\
            T &= \set{a,b} &\\
            S_{a2} &= Q_0 &\\
            P_{a2} &= \set{Q_0\to \lambda, Q_1\to Q_0a, Q_0\to Q_0b, Q_2\to Q_1a, Q_1\to Q_1b, Q_3\to Q_2a, Q_2\to Q_2b,&\\&\phantom{=\{ }\ Q_4\to Q_3a, Q_3\to Q_3b, Q_0\to Q_4a, Q_4\to Q_4b} &
        \end{flalign*}
        \item $G_b = (N_b, T, S_b, P_b)$\vspace{-2mm}
        \begin{flalign*}
            N_b &= \set{Q_0, Q_1, Q_2, Q_3} &\\
            T &= \set{a,b} &\\
            S_b &= Q_0 &\\
            P_b &= \set{Q_0\to \lambda, Q_1\to \lambda, Q_2\to \lambda, Q_2\to \lambda, Q_0\to bQ_0, Q_0\to aQ_1, Q_1\to aQ_1, Q_1\to bQ_2,&\\&\phantom{=\{ }\ Q_2\to aQ_3, Q_2\to bQ_0, Q_3\to aQ_3, Q_3\to bQ_3} &
        \end{flalign*}
        $G_{b2} = (N_{b2}, T, S_{b2}, P_{b2})$\vspace{-2mm}
        \begin{flalign*}
            N_{b2} &= \set{Q_0, Q_1, Q_2, Q_3, S} &\\
            T &= \set{a,b} &\\
            S_{b2} &= S &\\
            P_{b2} &= \set{Q_0\to \lambda, S\to Q_0, S\to Q_1, S\to Q_2, Q_0\to Q_0b, Q_1\to Q_0a, Q_1\to Q_1a, Q_2\to Q_1b,&\\&\phantom{=\{ }\ Q_3\to Q_2a, Q_0\to Q_2b, Q_3\to Q_3a, Q_3\to Q_3b} &
        \end{flalign*}
        \item $G_c = (N_c, T, S_c, P_c)$\vspace{-2mm}
        \begin{flalign*}
            N_c &= \set{Q_0, Q_1, Q_2, Q_3, Q_4} &\\
            T &= \set{a,b} &\\
            S_c &= Q_0 &\\
            P_c &= \set{Q_0\to \lambda, Q_1\to \lambda, Q_2\to \lambda, Q_2\to \lambda, Q_3\to \lambda, Q_0\to aQ_0, Q_0\to bQ_1, Q_1\to aQ_0, Q_1\to bQ_2,&\\&\phantom{=\{ }\ Q_2\to aQ_3, Q_2\to bQ_2, Q_3\to aQ_4, Q_3\to bQ_2, Q_4\to aQ_4, Q_4\to bQ_4} &
        \end{flalign*}
        $G_{c2} = (N_{c2}, T, S_c{c2}, P_{c2})$\vspace{-2mm}
        \begin{flalign*}
            N_{c2} &= \set{Q_0, Q_1, Q_2, Q_3, Q_4, S} &\\
            T &= \set{a,b} &\\
            S_{c2} &= Q_0 &\\
            P_{c2} &= \set{Q_0\to \lambda, S\to Q_0, S\to Q_1, S\to Q_2, S\to Q_3, Q_0\to Q_0a, Q_1\to Q_0b, Q_0\to Q_1b, Q_2\to Q_1b,&\\&\phantom{=\{ }\ Q_3\to Q_2a, Q_2\to Q_2b, Q_4\to Q_3a, Q_2\to Q_3b, Q_4\to Q_4a, Q_4\to Q_4b} &
        \end{flalign*}
    \end{enumerate}
\end{task}
\newpage
\begin{task}
    Pumping-Lemma: $z = 10^{n_L}1^{n_L}\$ 10^{n_L-1}10^{n_L}$. Damit $z_i = uv^iwx^iy \in L_{bin-bin+1}$,\\ muss $\abs{v} = \abs{x}$ und $w = 1^a \$ 1 0^b\text{ mit } a,b\in \mathbb{N}$. Da $\abs{vwx}\leq n_L \text{ und } \abs{vx} \geq 1$, muss $v = 1^c, x = 0^c\text{ mit } c \geq 1$.\\
    $\implies z_0 = 1 0^{n_L} 1^{n_L - c}\$ 1 0^{n_L - (c+1)} 1 0^{n_L}\notin L_{bin-bin+1} \implies L_{bin-bin+1}$ ist nicht kontextfrei.
\end{task}

\begin{task}
    \hfill\vspace{-5mm}
    \begin{enumerate}[label={(\alph*)}]
        \item Sei $L$ eine reguläre Sprache, dann ist $Sp(L)$ regulär. Dann existiert eine Zahl $n$, so dass für jedes Wort $z\in L \implies Sp(z) \in Sp(L)$ mit $\abs{z} = \abs{Sp(z)} \geq n$ eine Zerlegung existiert $Sp(z) = Sp(w) Sp(v) Sp(u) \implies z=uvw$ mit (1) $\abs{vw} = \abs{Sp(w) Sp(v)} \leq n$, (2) $\abs{v} = \abs{Sp(v)} \geq 1$ und (3) für alle $i\in \mathbb{N}$ gilt: $Sp(w)Sp(v)^i Sp(u) = Sp(wv^iu) \in Sp(L) \implies uv^iw\in L$.
        \item Beispiel aus der Vorlesung: $L = \set{a^i b^j c^k\mid i=0 \lor j=k}$, Nicht-Regularität kann mit Präfix-Version nicht nachgewiesen werden.\\
        Suffix-Version: $z = ab^{n_L}c^{n_L}, v = c^x, x\geq 1\text{, da } \abs{vw}\leq n_L.\implies z_0 = ab^{n_L}c^{n_L - x}\notin L$        
    \end{enumerate}
\end{task}
\end{document}
