\documentclass[11pt]{article}
\usepackage{geometry}
\geometry{a4paper, top=20mm, left=10mm, right=10mm, bottom=20mm}
\usepackage{graphicx}
\usepackage{amsmath,amssymb,amsthm}
\usepackage{amssymb}
\usepackage[utf8]{inputenc}
\usepackage{fancyhdr}
\usepackage{lastpage}
\usepackage{enumerate}
\usepackage{enumitem}
\usepackage{tikz}
\usetikzlibrary{shapes.geometric}
\usepackage{multicol}
\usepackage{subcaption}
\usepackage{ifthen}
%------------------------------------------ preamble
%----- fancyhdr
\fancyhead[R]{Übungsgruppe: 2 (Do 12-14)}
\fancyhead[C]{Name: Maurice Wenig}
\fancyhead[L]{Matrikelnummer: 178049}
\fancyfoot{}
\rfoot{Seite \thepage\ von \pageref{LastPage}}
\pagestyle{fancy}
%----- aufgaben
\newtheoremstyle{break}{}{5mm}{}{}{\bfseries}{}{0mm}
{\textbf{\thmname{#1}\thmnumber{ #2:} \thmnote{\textit{#3}}\newline}}
\theoremstyle{break}
\newtheorem{task}{Aufgabe}
%----- new commands
\newcommand{\Romannumeral}[1]{\MakeUppercase{\romannumeral #1}}
%------------------------------------------ main
\begin{document}
%----- title
\begin{center}
\Large{Automaten und Berechenbarkeit}\\
\large{1. Übungsserie}
\end{center}
%----- tasks
\begin{task}
\hfill\vspace{-5mm}
\begin{enumerate}
\item $\emptyset$
\item $\{ababaaaaabbaabababbabbbaabbabaababbbaabb\}$
\item $\{a,b\}^*$
\item $\{w\in\{a,b\}^*\mid w = Sp(w)\}$
\item $\{ab, ba, bb\}$
\item $\{a,b\}^{42}$
\item $\{xx\mid x\in\{a,b\}^*\}$
\item $\{w\in \{a,b\}^*\mid \left\vert w \right\vert_a = \left\vert w \right\vert_b\}$
\item $\{a\}^*$
\item $\{w\in\{a,b\}^*\mid \left\vert\left\vert w \right\vert_a - \left\vert w \right\vert_b\right\vert \leq 7 \}$
\end{enumerate}
\end{task}

\begin{task}
\hfill\vspace{-5mm}
\begin{enumerate}[label={(\alph*)}]
\item \hfill\vspace{-8mm} \begin{flalign*}
L_1 \times L_2 &= \{(ab),(abb),(abbb)\}&\\
L_1\cdot L_2 &= \{a,abb,abbb \}&\\
L_2\cdot L_1 &= \{ba,bab,bba,bbab \}&
\end{flalign*}
\item \hfill\vspace{-8mm} \begin{flalign*}
L_1^2 &= \{aa,aab,aba,abab\}&\\
L_1^3 &= \{aaa,aaab,aaba,aabab,abaa,abaab,ababa,ababab\}&\\
L_1^n &= \{a,ab\}^n&\\
L_1^* &= \{a,ab\}^*&\\
L_2^2 &= \{bb,bbb,bbbb\}&\\
L_2^3 &= \{bbb,bbbb,bbbbb,bbbbbb\}&\\
L_2^n &= \{w\mid n \leq \left\vert w \right\vert_b = \left\vert w \right\vert \leq 2n \}&\\
L_2^* &= \{b\}^*&
\end{flalign*}
\end{enumerate}
\end{task}

\begin{task}
\hfill\vspace{-5mm}
\begin{enumerate}[label={(\alph*)}]
\item $L_1\cdot (L_2\cup L_3)\supseteq L_1\cdot L_2 \cup L_1\cdot L_3$
\begin{flalign*}
L_1\cdot (L_2\cup L_3) &= \{w\in\Sigma^*\mid\exists u,v\in\Sigma^*:(w=u\cdot v \land u\in L_1 \land (v\in L_2\lor v\in L_3)) \}&\\
&= \{w\in\Sigma^*\mid \exists u,v\in\Sigma^*: \left(w=u\cdot v \land u\in L_1\land v\in L_2\lor w=u\cdot v \land u\in L_1\land v\in L_3\right)\}&
\end{flalign*}
\begin{flalign*}
L_1\cdot L_2 \cup L_1\cdot L_3 &= \{w\in\Sigma^*\mid \exists u,v\in\Sigma^*: (w=u\cdot v \land u\in L_1\land v\in L_2)\}\\&\ \hspace{4mm}\cup \{w\in\Sigma^*\mid \exists u,v\in\Sigma^*: (w=u\cdot v \land u\in L_1\land v\in L_3)\}&\\
&=\{w\in\Sigma^*\mid \exists u,v\in\Sigma^*: \left(w=u\cdot v \land u\in L_1\land v\in L_2\right)\lor\exists u,v\in\Sigma^*: \left(w=u\cdot v \land u\in L_1\dots\right)\}&\\
\end{flalign*}
Da $\exists u,v\in\Sigma^*: \left(w=u\cdot v \land u\in L_1\land v\in L_2\right)\lor\exists u,v\in\Sigma^*: \left(w=u\cdot v \land u\in L_1\land v\in L_3\right)\\\implies \exists u,v\in\Sigma^*: \left(w=u\cdot v \land u\in L_1\land v\in L_2\lor w=u\cdot v \land u\in L_1\land v\in L_3\right)$,\\gilt $L_1\cdot (L_2\cup L_3)\supseteq L_1\cdot L_2 \cup L_1\cdot L_3$.

\item $L_1\cdot (L_2\cap L_3)\subseteq L_1\cdot L_2 \cap L_1\cdot L_3$
\begin{flalign*}
L_1\cdot (L_2\cap L_3) &= \{w\in\Sigma^*\mid\exists u,v\in\Sigma^*:(w=u\cdot v \land u\in L_1 \land v\in L_2\land v\in L_3) \}&
\end{flalign*}
\begin{flalign*}
L_1\cdot L_2 \cap L_1\cdot L_3 &= \{w\in\Sigma^*\mid \exists u,v\in\Sigma^*: (w=u\cdot v \land u\in L_1\land v\in L_2)\}\\&\ \hspace{4mm}\cap \{w\in\Sigma^*\mid \exists u,v\in\Sigma^*: (w=u\cdot v \land u\in L_1\land v\in L_3)\}&\\
&= \{w\in\Sigma^*\mid\exists u,v\in\Sigma^*:(w=u\cdot v \land u\in L_1 \land v\in L_2)\land\exists u,v\in\Sigma^*:(w=u\cdot v \land u\in L_1 \dots) \}&
\end{flalign*}
Da $\exists u,v\in\Sigma^*:(w=u\cdot v \land u\in L_1 \land v\in L_2\land v\in L_3)\\
\implies \exists u,v\in\Sigma^*:(w=u\cdot v \land u\in L_1 \land v\in L_2)\land\exists u,v\in\Sigma^*:(w=u\cdot v \land u\in L_1 \land v\in L_3)$,\\
gilt $L_1\cdot (L_2\cap L_3)\subseteq L_1\cdot L_2 \cap L_1\cdot L_3$.
\end{enumerate}
\end{task}

\begin{task}
\hfill\vspace{-5mm}
\begin{enumerate}[label={(\alph*)}]
\item Nein, Gegenbeispiel: $L_1 = \{|,||\},\ L_2 = \{|\},\ L_3 = \{||\}$
\item Ja, da $\Sigma_1$ und $\Sigma_2$ disjunkt sind, muss $\exists a\in\Sigma_1^*:\exists b\in\Sigma_2^*:(w=ab)$ mit $w\in L_A\cdot L_B$ und $L_A\in\Sigma_1^*,\ L_B\in\Sigma_2^*$. Wenn $w = ab \in L_1\cdot L_2 \land w = ab \in L_1\cdot L_3$, dann $a\in L_1 \land b\in L_2 \land b\in L_3 \implies b\in L_2\cap L_3 \implies w\in L_1\cdot (L_2\cap L_3)$.
Also ist $L_1\cdot (L_2\cap L_3)\supseteq L_1\cdot L_2 \cap L_1\cdot L_3\stackrel{\mbox{\tiny 3.(b)}}{\implies}L_1\cdot (L_2\cap L_3) = L_1\cdot L_2 \cap L_1\cdot L_3$
\end{enumerate}
\end{task}

\begin{task}
Hilfssatz:\\
Sei $y_1,y_2,\dots,y_n\in \{1,2\}$,  die dyadische Darstellung einer natürlichen Zahl $a>0$. Dann ist $x_1,x_2,\dots,x_m\in \{1,2\}$ mit $m\geq n$ und $\exists k\in[1,m]:x_{m-k}\neq y_{n-k}$ ($y$ mit nicht-positivem Index wird als 0 angenommen) die dyadische Darstellung einer natürlichen Zahl $b\neq a$:\\
$$b-a = \sum\limits_{i=1}^{m}x_i\cdot 2^{m-i}-\sum\limits_{i=1}^{n}y_i\cdot 2^{n-i}=\underbrace{\sum\limits_{i=1}^{m-n}x_i\cdot 2^{m-i}}_{\neq 0,\ \geq 2^n {}^{(*)}} + \underbrace{\sum\limits_{i=1}^{n}(x_{i+m-n}-y_i)\cdot 2^{n-i}}_{\neq 0 ,\ >(-2^n){}^{(**)}}\neq 0$$\\
(*) $\neq 0,\ \geq 2^n$, falls $k>n$\\
(**) $\neq 0$, falls $k\leq n$; $>(-2^n)$ da $(x_{i+m-n}-y_i)\in\{1,0,-1\}$\vspace{3mm}\\
\begin{tabular}{rl}
IA:&$n=1$, 1 hat genau eine dyadische dyadische Darstellung $1_{dya}$\\
IV:&für $n=k$ gilt: $k_{dya}$ hat genau eine dyadische Darstellung\\
IB:&für $n=k+1$ gilt: $(k+1)$ hat genau eine dyadische Darstellung\\
IS:&$k\rightarrow k+1$\\
&Da $k$ eine dyadische Darstellung $k_{dya}$ hat $k+1$ die dyadische Darstellung $k_{dya}+1$.\\
&Alle anderen dyadischen Folgen stellen nicht $k+1$ dar (Hilfssatz), wodurch $k+1$\\
&genau eine dyadische Darstellung hat.
\end{tabular}\vspace{3mm}
\end{task}
\end{document}
