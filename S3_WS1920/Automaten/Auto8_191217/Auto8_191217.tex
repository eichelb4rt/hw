\documentclass[11pt]{article}
\usepackage{geometry}
\geometry{a4paper, top=20mm, left=10mm, right=10mm, bottom=20mm}
\usepackage{graphicx}
\usepackage{amsmath,amssymb,amsthm}
\usepackage{amssymb}
\usepackage[utf8]{inputenc}
\usepackage{fancyhdr}
\usepackage{lastpage}
\usepackage{enumerate}
\usepackage{enumitem}
\usepackage{tikz}
\usetikzlibrary{shapes.geometric}
\usepackage{multicol}
\usepackage{subcaption}
\usepackage{ifthen}
%------------------------------------------ preamble
%----- fancyhdr
\fancyhead[R]{Übungsgruppe: 2 (Do 12-14)}
\fancyhead[C]{Name: Maurice Wenig}
\fancyhead[L]{Matrikelnummer: 178049}
\fancyfoot{}
\rfoot{Seite \thepage\ von \pageref{LastPage}}
\pagestyle{fancy}
%----- aufgaben
\newtheoremstyle{break}{}{5mm}{}{}{\bfseries}{}{0mm}
{\textbf{\thmname{#1}\thmnumber{ #2:} \thmnote{\textit{#3}}\newline}}
\theoremstyle{break}
\newtheorem{task}{Aufgabe}
%----- new commands
\newcommand{\Romannumeral}[1]{\MakeUppercase{\romannumeral #1}}
\newcommand{\notiff}{\mathrel{{\ooalign{\hidewidth$\not\phantom{"}$\hidewidth\cr$\iff$}}}}
\newcommand{\set}[1]{\ensuremath{\{#1\}}}
\newcommand{\abs}[1]{\ensuremath{\left\vert #1 \right\vert}}
%----- tikz automata
\usetikzlibrary{arrows, automata, positioning}
%------------------------------------------ main
\begin{document}
%----- title
\begin{center}
\Large{Automaten und Berechenbarkeit}\\
\large{8. Übungsserie}
\end{center}
%----- tasks
\begin{task}
    $G_4 = (N, T, S, P)$\vspace{-2mm}
    \begin{flalign*}
        N &= \set{S, L, R} &\\
        T &= \set{a,b,c,d} &\\
        S &= S &\\
        P &= \set{S\to \lambda, S\to abcd, S\to Rabcd, Ra\to aR, Rb\to bR, Rc\to cR, dL\to Ld, cL\to Lc, bL\to Lb,&\\&\phantom{=}\ aRb\to aabbR, cRd\to ccddL, cRd\to Lccdd, aLb\to Raabb, cRd\to ccdd, aLb\to aabb } &
    \end{flalign*}
    $\implies L$ ist kontextsensitiv\vspace{3mm}\\
    Pumping-Lemma: $z = a^{n_L} b^{n_L} c^{n_L} d^{n_L}, \abs{vwx} \leq n_L \implies \text{$vwx$ besteht aus maximal 2 unterschiedlichen Buchstaben}\\\implies uv^iwx^iy \text{ mit $i\neq 1$ ändert die Anzahl von maximal 2 unterschiedlichen Buchstaben}\implies z\notin L_4$\vspace{3mm}\\
    $\implies L$ ist nicht kontextfrei $\implies L\in CH(1)$
\end{task}

\begin{task}
    Intuition: Um festzustellen, ob ein beliebiges $w\in \overline{L_3}$, sollte man überprüfen, ob $w\in L_3$ $\implies \overline{L_3}$ ist kontextsensitiv.\\
    Wikipedia: $\mathcal{CSL}$ ist unter Komplementbildung abgeschlossen $\implies \overline{L_3}$ ist kontextsensitiv.\\
    Pumping-Lemma: Funktioniert, $\overline{L_3}$ könnte kontextfrei sein.
\end{task}

\begin{task}
    Nein, Pumping-Lemma: $z = a^{n_L} b^{n_L^2}, \abs{vwx}\leq n_L\implies \abs{vx}_b \leq n_L\\\implies n_L^2 + \abs{vx}_b \leq n_L^2 + n_L < n_L^2 + 2 n_L  + 1 = (n_L + 1)^2 \stackrel{\mbox\tiny{\abs{vx}_b > 0}}{\implies} z_L^2 < \abs{z_2}_b < (z_L + 1)^2\implies z_2\notin L$\\
    falls $\abs{vx}_b = 0$ dann ist $vx = a^\alpha, \alpha \geq 1\implies z_2 \notin L$
\end{task}

\begin{task}
    \hfill\vspace{-5mm}
    \begin{enumerate}[label={(\alph*)}]
        \item $G_{Sp-bin} = (N, T, S, P)$\vspace{-2mm}
        \begin{flalign*}
            N &= \set{S, X} &\\
            T &= \set{0,1, \$} &\\
            S &= S &\\
            P &= \set{S\to 0\$0, S\to X, X\to 0X0, X\to 1X1, X\to 1\$1} &
        \end{flalign*} (ist kontextfrei)
        \item \hfill\vspace{-5mm}\begin{enumerate}[align=left, label={Fall \arabic*:}]
            \item $\mathrm{Sp}(\mathrm{bin}(n)) = \mathrm{bin}(n) = 1^n \implies \mathrm{bin}(n+1) = 10^n$
            \item $\mathrm{bin}(n) = r 0 1^n, n\in \mathbb{N}, r\in \set{1}\cdot \Sigma^*\implies \mathrm{Sp}(\mathrm{bin}(n)) = 1^n 0 \mathrm{Sp}(r), \mathrm{bin}(n+1) = r 1 0^n$
        \end{enumerate}
        $\implies G_{Sp-bin+1} = (N, T, S, P)$\vspace{-2mm}
        \begin{flalign*}
            N &= \set{S, X, Y} &\\
            T &= \set{0,1, \$} &\\
            S &= S &\\
            P &= \set{S\to 0\$1, S\to X, X\to 1X0, X\stackrel{\mbox{\tiny{Fall 1}}}{\to}1\$10, X\stackrel{\mbox{\tiny{Fall 2}}}{\to} 0Y1, Y\to 1Y1, Y\to 0Y0, Y\to 1\$1 } &
        \end{flalign*} (ist kontextfrei)
    \end{enumerate}
\end{task}
\end{document}
