\documentclass[11pt]{article}
\usepackage{geometry}
\geometry{a4paper, top=20mm, left=10mm, right=10mm, bottom=20mm}
\usepackage{graphicx}
\usepackage{amsmath,amssymb,amsthm}
\usepackage{amssymb}
\usepackage[utf8]{inputenc}
\usepackage{fancyhdr}
\usepackage{lastpage}
\usepackage{enumerate}
\usepackage{enumitem}
\usepackage{tikz}
\usetikzlibrary{shapes.geometric}
\usepackage{multicol}
\usepackage{subcaption}
\usepackage{ifthen}
%------------------------------------------ preamble
%----- fancyhdr
\fancyhead[R]{Übungsgruppe: 2 (Do 12-14)}
\fancyhead[C]{Name: Maurice Wenig}
\fancyhead[L]{Matrikelnummer: 178049}
\fancyfoot{}
\rfoot{Seite \thepage\ von \pageref{LastPage}}
\pagestyle{fancy}
%----- aufgaben
\newtheoremstyle{break}{}{5mm}{}{}{\bfseries}{}{0mm}
{\textbf{\thmname{#1}\thmnumber{ #2:} \thmnote{\textit{#3}}\newline}}
\theoremstyle{break}
\newtheorem{task}{Aufgabe}
%----- new commands
\newcommand{\Romannumeral}[1]{\MakeUppercase{\romannumeral #1}}
\newcommand{\notiff}{\mathrel{{\ooalign{\hidewidth$\not\phantom{"}$\hidewidth\cr$\iff$}}}}
%----- tikz automata
\usetikzlibrary{arrows, automata, positioning}
%------------------------------------------ main
\begin{document}
%----- title
\begin{center}
\Large{Automaten und Berechenbarkeit}\\
\large{6. Übungsserie}
\end{center}
%----- tasks
\begin{task}
    \hfill\vspace{-5mm}
    \begin{enumerate}[label={(\alph*)}]
        \item Nein, Pumping Lemma: $n\in \mathbb{N},\ z = \#a^{2^n}\#\\ uv = \#a^m,
        v = 
        \begin{cases}
            a^r & uv^2 w = \#a^{2^n + r}\#,\ r < 2^n\implies 2^n < 2^n + r < 2^{n+1}\implies \underline{\underline{uv^2w \notin L}}\\
            \#a^r & uv^2w = \#a^x\#a^y\#a^z \notin L
        \end{cases}$
        \item $F = \{ \{ \#a^m\}\mid m\in \mathbb{N} \}\cup \{\{\lambda\}\}\cup (\Sigma^*\setminus\{ \{ \#a^m\}\mid m\in \mathbb{N} \}\cup \{\{\lambda\}\})$ , denn für $m\neq n \in \mathbb{N}$ gilt \\$\exists_{d\in \mathbb{N}}: \#a^{m+d}\#\in L \land \#a^{n+d}\#\notin L$. Da der Schnitt zweier Äquivalenzklassen leer ist und die Vereinigung aller Äquivalenzklassen $\Sigma^*$ darstellt, ist $F$ eine Zerlegung von $\Sigma^*$.
    \end{enumerate}
\end{task}

\begin{task}
    \hfill\vspace{-5mm}
    \begin{enumerate}[label={(\alph*)}]
        \item $G = (N,T,S,P)$
        \begin{flalign*}
            N &= \{X\}&\\
            T &= \{a,b\}&\\
            S &= \{X\}&\\
            P &= \{ X\rightarrow \lambda,\ X\rightarrow XX,\ X\rightarrow aXb,\ X\rightarrow bXa\} &
        \end{flalign*}
        $L(G)\subseteq L_{2equal}$ (trivial)\\
        $L_{2equal}\subseteq L(G)$:\vspace{3mm}\\
        \begin{tabular}{rl}
            IA:&$n = |v| = 0$, $v\in L_{2equal}, v\in L(G)$\\
            IV:&für $n = |v|, v\in L_{2equal}$ gilt: $v\in L(G)$\\
            IB:&für $n+2 = |uvw|, uvw\in L_{2equal}$ gilt: $uvw\in L(G)$\\
            IS:&$v\rightarrow uvw$\\
            &$uvw = \begin{cases}
                abv & X\rightarrow^* v\implies X\rightarrow XX \rightarrow abX \rightarrow^* abv\\
                bav & X\rightarrow^* v\implies X\rightarrow XX \rightarrow baX \rightarrow^* bav\\
                vab & X\rightarrow^* v\implies X\rightarrow XX \rightarrow Xab \rightarrow^* vab\\
                vba & X\rightarrow^* v\implies X\rightarrow XX \rightarrow Xba \rightarrow^* vba\\
                avb & X\rightarrow^* v\implies X\rightarrow aXb \rightarrow^* avb\\
                bva & X\rightarrow^* v\implies X\rightarrow bXa \rightarrow^* bva
            \end{cases}\implies \underline{\underline{uvw \in L(G)}}$
            \end{tabular}\vspace{3mm}\\
            $\implies L(G) = L_{2equal}$
            \item $X\rightarrow XX\rightarrow XaXb\rightarrow abaXb\rightarrow ababXab\rightarrow ababbaab\\
            X\rightarrow XX\rightarrow XXXX\rightarrow abbaabba$
            \item $|F_L(L_{2equal})| = |Q|\ (\text{Differenz der Anzahl von } a \text{ und } b) = \aleph_0\implies L_{2equal}$ ist nicht regulär
    \end{enumerate}
\end{task}

\begin{task}
    \hfill
\end{task}
\end{document}
