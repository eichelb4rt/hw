\documentclass[11pt]{article}
\usepackage{geometry}
\geometry{a4paper, top=20mm, left=10mm, right=10mm, bottom=20mm}
\usepackage{graphicx}
\usepackage{amsmath,amssymb,amsthm}
\usepackage{amssymb}
\usepackage[utf8]{inputenc}
\usepackage{fancyhdr}
\usepackage{lastpage}
\usepackage{enumerate}
\usepackage{enumitem}
\usepackage{color}
\usepackage{scalerel}
\usepackage{tabularx}
%------------------------------------------ preamble
%----- fancyhdr
\fancyhead[R]{Übungsgruppe: B (Mo 10-12)}
\fancyhead[C]{Name: Maurice Wenig}
\fancyhead[L]{Matrikelnummer: 178049}
\fancyfoot{}
\rfoot{Seite \thepage\ von \pageref{LastPage}}
\pagestyle{fancy}
%----- aufgaben
\newtheoremstyle{break}{}{5mm}{}{}{\bfseries}{}{0mm}
{\textbf{\thmname{#1}\thmnumber{ \hw.#2:} \thmnote{\textit{#3}}\newline}}
\theoremstyle{break}
\newtheorem{task}{Aufgabe}
%----- new environments
\newcounter{premises}
\newenvironment{argument}
{
    \setcounter{premises}{1}    % set the counter to 1
    \def\P{
        $P_\arabic{premises}$ & 
        \stepcounter{premises}
    }
    \def\S{\hline $S$ &}
    \def\K{\hline $K$ &}
    % begin of the actual environment
    \flushleft
    \tabularx{0.8\textwidth}{cX}
}
{
    \endtabularx
}
%----- defs
\def\notiff{\mathrel{{\ooalign{\hidewidth$\not\phantom{"}$\hidewidth\cr$\iff$}}}}
\def\R{\ensuremath{\mathbb{R}}}
\def\1{\ensuremath{\mathbb{1}}}
\def\ndy{
    \textcolor{red} {\hfill not done yet!}
    \reversemarginpar
    \marginpar{\raggedleft\textcolor{red}{\rule{2mm}{2mm}}}
}
%----- homework number
\newcommand{\hw}{2}
%------------------------------------------ main
\begin{document}
%----- title
\begin{center}
\Large{Logik und Argumentationslehre}\\
\large{\hw. Übungsserie}
\end{center}
%----- tasks
\begin{task}
    Zwischenargument 1: nicht gültig
    \begin{argument}
        \P Es ist keine unfehlbare Erkenntnisquelle bekannt ("Welche sollte es denn sein?").\\
        \K Wir verfügen nicht über eine unfehlbare Erkenntnisquelle.
    \end{argument}\vspace*{3mm}\\
    Zwischenargument 2: gültig, nicht schlüssig
    \begin{argument}
        \P Wenn wir sicheres Wissen über die Außenwelt haben, dann verfügen wir auch über eine unfehlbare Erkenntnisquelle.\\
        \P Wir verfügen nicht über eine unfehlbare Erkenntnisquelle.\\
        \S MT($P_1,P_2$)\\
        \K Wir haben kein sicheres Wissen über die Außenwelt.
    \end{argument}
\end{task}
\end{document}