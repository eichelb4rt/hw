\documentclass[11pt]{article}
\usepackage{geometry}
\geometry{a4paper, top=20mm, left=10mm, right=10mm, bottom=20mm}
\usepackage{graphicx}
\usepackage{amsmath,amssymb,amsthm}
\usepackage{amssymb}
\usepackage[utf8]{inputenc}
\usepackage{fancyhdr}
\usepackage{lastpage}
\usepackage{enumerate}
\usepackage{enumitem}
\usepackage{multicol}
\usepackage{subcaption}
\usepackage{ifthen}
\usepackage{listings}
\usepackage{color}
\usepackage{scalerel}
%------------------------------------------ preamble
%----- fancyhdr
\fancyhead[R]{Übungsgruppe: B (Mo 10-12)}
\fancyhead[C]{Name: Maurice Wenig}
\fancyhead[L]{Matrikelnummer: 178049}
\fancyfoot{}
\rfoot{Seite \thepage\ von \pageref{LastPage}}
\pagestyle{fancy}
%----- aufgaben
\newtheoremstyle{break}{}{5mm}{}{}{\bfseries}{}{0mm}
{\textbf{\thmname{#1}\thmnumber{ \hw.#2:} \thmnote{\textit{#3}}\newline}}
\theoremstyle{break}
\newtheorem{task}{Aufgabe}
%----- listings
\definecolor{mygreen}{rgb}{0,0.6,0}
\definecolor{mygray}{rgb}{0.5,0.5,0.5}
\definecolor{mymauve}{rgb}{0.58,0,0.82}
\lstset{ %
backgroundcolor=\color{white}, 
% choose the background color; you must add
basicstyle=\footnotesize, 
% the size of the fonts that are used for the code
breakatwhitespace=false, 
% sets if automatic breaks should only happen at whitespace
breaklines=true,
% sets automatic line breaking
captionpos=b,
% sets the caption-position to bottom
commentstyle=\color{mygreen},
% comment style
deletekeywords={...},
% if you want to delete keywords from the given language
extendedchars=true,
% lets you use non-ASCII characters; for 8-bits encodings only   
% adds a frame around the code
keepspaces=true, 
% keeps spaces in text, useful for keeping indentation of code
keywordstyle=\color{blue}, 
% keyword style
language=java, 
% the language of the code
otherkeywords={@Override},
% if you want to add more keywords to the set
numbers=left, 
% where to put the line-numbers; possible values(none, left, right)
numbersep=5pt, 
% how far the line-numbers are from the code
numberstyle=\tiny\color{mygray}, % the style that is used for the line-numbers
% if not set, frame-color can change on line-breaks to text-color
showspaces=false, 
% show spaces everywhere adding particular underscores;
showstringspaces=false, 
% underline spaces within strings only
showtabs=false, 
% show tabs within strings adding particular underscores
stepnumber=1, 
% the step between two line-numbers.
stringstyle=\color{mymauve}, 
% string literal style
tabsize=2,     
% sets default tabsize to 2 spaces
% show the filename of files included with \lstinputlisting;
emph={@Override},
emphstyle={\color{Goldenrod}}
}
%----- new commands
\newcommand{\Romannumeral}[1]{\MakeUppercase{\romannumeral #1}}
\newcommand{\set}[1]{\ensuremath{\{#1\}}}
\newcommand{\abs}[1]{\ensuremath{\left\vert #1 \right\vert}}
\newcommand{\norm}[1]{\ensuremath{\left\| #1 \right\|}}
\newcommand{\skal}[2]{\ensuremath{\left\langle #1 | #2 \right\rangle}}
\newcommand{\script}[1]{
    skripte/aufgabe#1.py
    \lstinputlisting{skripte/aufgabe#1.py}
}
%----- defs
\def\notiff{\mathrel{{\ooalign{\hidewidth$\not\phantom{"}$\hidewidth\cr$\iff$}}}}
\def\R{\ensuremath{\mathbb{R}}}
\def\1{\ensuremath{\mathbb{1}}}
\def\ndy{
    \textcolor{red} {\hfill not done yet!}
    \reversemarginpar
    \marginpar{\raggedleft\textcolor{red}{\rule{2mm}{2mm}}}
}
%----- homework number
\newcommand{\hw}{1}
%------------------------------------------ main
\begin{document}
%----- title
\begin{center}
\Large{Logik und Argumentationslehre}\\
\large{\hw. Übungsserie}
\end{center}
%----- tasks
\begin{task}
    Annahme: unsere Axiome sind wahr.\\
    Zwischenargument: gültig, schlüssig
    \begin{enumerate}
        \item[$\text{P}_1$:] Angela Merkel besuchte am Abend des Mauerfalls eine Sauna.
        \item[$\text{P}_2$:] Der Mauerfall war 1989.\vspace*{3mm}\\
        (Man muss geboren wurden sein, um eine Sauna zu besuchen.)
        \item[K] Angela Merkel wurde vor/in 1989 geboren.
    \end{enumerate}
    Zwischenargument: gültig, schlüssig
    \begin{enumerate}
        \item[$\text{P}_1$:] Angela Merkel wurde vor/in 1989 geboren.
        \item[$\text{P}_2$:] Antonio Rüdiger wurde 1993 geboren.\vspace*{3mm}\\
        (1989 war vor 1993.) 
        \item[K] Angela Merkel wurde vor Antonia Rüger geboren.
    \end{enumerate}
    Zwischenargument: gültig, schlüssig
    \begin{enumerate}
        \item[$\text{P}_1$:] Angela Merkel wurde vor Antonia Rüger geboren.\vspace*{3mm}\\
        (Eltern werden vor ihrem Kind geboren.)
        \item[K] Antonia Rüger ist nicht Angela Merkels Vater.
    \end{enumerate}
    Argument: gültig, schlüssig
    \begin{enumerate}
        \item[$\text{P}_1$:] Antonio Rüger ist nicht Angela Merkels Vater.\vspace*{3mm}\\
        (Wer Fakten nicht anerkennt, hat nicht alle Tassen im Schrank.)
        \item[K] Wer behauptet, dass Antonio Rüdiger der Vater von Angela Merkel ist, der hat nicht alle Tassen im Schrank.
    \end{enumerate}
\end{task}
\end{document}